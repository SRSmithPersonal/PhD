\chapter{Conclusion}
\label{chp:Conclusion}

We designed and constructed a novel time domain THz ellipsometer. We developed our own THz polarization optics, which in itselves is novel within the THz field. Two novel THz ellipsometric data extraction techniques, the single-layer isotropic technique and the two layer ellipsometric technique, were developed. A THz simulation method was pioneered, which was then used to test the aforementioned data extraction techniques, as well as two other data extraction techniques: the bulk isotropic technique and the self-reference method, which were complimentary to our system.

The ellipsometric system was used to obtain time-domain measurements of several different samples, with the complex refractive index extracted from these measurements using the appropriate data extraction technique. From these measurements we found that the system and data extraction techniques worked for a bulk isotropic sample and isotropic samples that are not highly transparent. For transparent samples it was found that the extraction technique performed well for extracting the real refractive index, but is limited in how well it can extract the very small extinction coefficient. This is in part due to the accuracy to which we can measure the angle of incidence. For ethanol-water mixtures of different concentration, both the two-layer ellipsometric method and self-reference method showed that higher ethanol concentrations lead to reduced real refractive indexes and extinction coefficients extracted for the sample. For liquid water the self-reference method produced results similar to those found in literature, but with a slightly lower imaginary dielectric constant and a slightly higher real dielectric constant than expected. 
The two layer ellipsometric method produced a real dielectric constant close to the expected value for water in the range of $0.5\,$THz to $0.9\,$THz. Beyond $0.9\,$THz the real dielectric constant rapidly diverged from the values found in literature.
The extracted imaginary dielectric constant, in the range of $0.5\,$THz to $1.2\,$THz, followed a different trend from those found in literature, however the mean value agreed fairly well.

These results show the power of the data extraction techniques with the setup. Together they exhibit the ability to extract optical information for a multitude of optically isotropic samples. The system can be improved with the future development of THz polarization optics, such as half-wave plates, which would allow for the setup to be more compact. Such a compact setup would be easy to manufacture as a turn-key system. Further expansion and optimization of the data extraction techniques can make this a sought after tool in many industries, such as semi-conductive material production and drug research.

\section{Future Work}
\label{sec: Future}

Water vapour is a major issue for terahertz spectroscopy in general. We found that water vapour absorption resonances are still present in our measured data after the chamber was purged with nitrogen. A possible solution to this situation would be to design a smaller, airtight chamber, and perform measurements in vacuum.
\paragraph{}
%It would be preferable to reduce the size of the flushing chamber, as this would allow for a faster and more stable flushing process. Currently the chamber is $~160L$, which can take between $15$ and $20$ minutes to purge to below $0.1\,\%$ humidity with dry nitrogen. This flushing process currently expends an exorbitant amount of nitrogen gas and can be somewhat unstable. On proposed solution is to use a smaller, segmented chamber system. This can drastically reduce the volume of the flushing chamber, which in turn would lead to a faster, more stable purge.
The production of a second cuvette, with a different chamber material, might be of use, as this can help determine the cause of the error present in the two-layer measurements. By manufacturing the cuvette from a material with a similar refractive index to that of the window, any reflections from the interface between the window and cuvette will be greatly minimized. If the cuvette body and window were made from the same material, these reflections can be completely eliminated. A candidate for this would be polyethylene, as it has near frequency independent characteristics in the THz region, and it is possible to etch or cast a unibody cuvette from polyethylene, if the proper facilities are available.
\paragraph{}
Improving the positioning accuracy of the sample can improve the overall functionality of the system. A mount which can consistently mount the sample at a given angle with an accuracy of $0.002^{\circ}$ would contribute considerably to to measurement and analysis of transparent materials, as the measured data has shown.
%The development of a more stable sample mount will greatly assist when multiple samples need to be investigated.
\paragraph{}
Redesigning the ellipsometric setup with the goal of changing the angle of incidence on the sample to $45^{\circ}$ might be of use. At $45^{\circ}$ incidence, a lower accuracy in general is expected from the data extraction techniques, as the ratio between the reflection coefficients will be closer to one. But the reflection coefficient also vary slower with the angle of incidence, as this is further from the Brewster angle for most samples. This, in turn, results in the system being less affected by misalignment. The construction and evaluation of our time domain THz eliipsometer therefore highlighted a number of design improvements which can be implemented to increase both the ease of use as well as the accuracy of the extracted complex refractive index of a variety of samples.
%----------------------------------------------------------------------------
\endinput