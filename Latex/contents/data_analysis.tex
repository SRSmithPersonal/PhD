\chapter{Data Analysis}
\label{chp:Analysis}

Currently there is a distinct lack of data extraction techniques developed for terahertz time-domain ellipsometry. These techniques are required for extracting material information contained within measured data. The techniques discussed in this chapter focus predominantly on the extraction of the complex refractive index for a given sample from measured data.Three different techniques will be discussed, each with a specific usage case.

\section{Bulk isotropic model}
\label{sec:BIM}
Bulk isotropic materials are optically isotropic dielectric materials, hence not causing depolarization, that have a thickness and optical density that does not allow for measurable internal reflections. Accordingly, for this model only first surface reflections will be considered.

\begin{figure}[H]
\begin{center}
	 \includegraphics[scale=0.8]{figs/BulkDiag.png}
	 \caption{Diagram depicting light matter interaction with a bulk isotropic sample. Only surface reflections are measurable, as internal reflections are not measurable.}
   \label{fig:BulkDiag}
\end{center}
\end{figure}

\begin{figure}[H]
\begin{center}
	 \includegraphics[scale=0.8]{figs/Time-325-1000.png}
	 \caption{Simulated terahertz electric field $E_{0}$ in time and the s- and p-polarized components of said electric field after being reflected from a bulk isotropic sample with a real refractive index of $3.25$ and absorption coefficient of $1000\,\mbox{cm}^{-1}$}
   \label{fig:BulkTime}
\end{center}
\end{figure}

Let us consider the electric field reflected from a bulk isotropic sample. The observed reflected electric field measured in time for the s- and p-polarization can be described as follows

\begin{eqnarray}
E_{s}(t) &=& r_{s}(f)E_{0}(t) \label{eq:Es-bulk Time}\\
E_{p}(t) &=& r_{p}(f)E_{0}(t) \label{eq:Ep-bulk Time}
\end{eqnarray}

where $r_{s}(f)$ and $r_{p}(f)$ are the frequency dependent s- and p-reflection coefficients (equation \ref{eq:FresnelRP} and \ref{eq:FresnelRS}) and $E_{0}$ is the incident electric field.

It should be noted that $r_{s}(f)$ and $r_{p}(f)$ are frequency dependent, hence it is convenient to work in the frequency domain. By performing a fast Fourier transform (FFT) on the data it is converted from the time to the frequency domain, hence the electric field components are rewritten as

\begin{eqnarray}
E_{s}(f) &=& r_{s}(f)E_{0}(f) \label{eq:Es-bulk Frequency}\\
E_{p}(f) &=& r_{p}(f)E_{0}(f) \label{eq:Ep-bulk Frequency}
\end{eqnarray}

Using standard ellipsometric data analysis, it is possible to extract the complex refractive index from this data.\cite{Tompkins-2005}

\begin{eqnarray}
P(f) &=& \frac{E_{p}(f)}{E_{s}(f)}\\
\widetilde{\epsilon}(f) &=& \widetilde{n}_{0}\sin^{2}{\theta}\left[1 + \left(\frac{1-P(f)}{1+P(f)}\right)^{2}\tan^{2}{\theta}\right]
\label{eq:Ellips}\\
\widetilde{n}(f) &=& \sqrt{\widetilde{\epsilon}(f)}
\end{eqnarray}

The angle of incidences is given by $\theta$ in equation \ref{eq:Ellips}, $\widetilde{n}_{0}$ is the complex refractive index of the material surrounding the sample, $\widetilde{\epsilon}(f)$ is the frequency dependent complex dielectric constant of the sample and $\widetilde{n}(f)$ is the frequency dependent complex refractive index of the material.

\begin{figure}[H]
                \begin{center}$
								\begin{array}{cc}
                \includegraphics[scale=0.5]{figs/n-325-1000.png}&
                \includegraphics[scale=0.5]{figs/k-325-1000.png}
								\end{array}$
								\end{center}
	\caption{The complex refractive index extracted from the data presented in figure \ref{fig:BulkTime} via the bulk isotropic data extraction technique}
	\label{fig:BulkExt}
\end{figure}

\section{Single layer isotropic model}
\label{sec:SLM}

Single layer isotropic samples are optically isotropic dielectric materials. These samples have a thickness and optical density that allow for observable internal reflections. This model expands on the model in section \ref{sec:BIM} by incorporating said internal reflections.

\begin{figure}[H]
\begin{center}
	 \includegraphics[scale=0.8]{figs/SingleDiag.png}
	 \caption{Diagram depicting light matter interaction with a single layer isotropic sample. Both surface reflections and internal reflections are measurable.}
   \label{fig:SingleDiag}
\end{center}
\end{figure}

\begin{figure}[H]
\begin{center}
	 \includegraphics[scale=0.8]{figs/Time-325-20.png}
	 \caption{Simulated terahertz electric field $E_{0}$ in time and the s- and p-polarized components of said electric field after being reflected from a single layer isotropic sample with a real refractive index of $3.25$ and absorption coefficient of $20\,\mbox{cm}^{-1}$}
   \label{fig:SingTime}
\end{center}
\end{figure}

\subsection{Transfer Function}
\label{sub: transp}
Transfer functions describe how light travels through a sample. 

When taking a measurement in time of a single layer isotropic system an initial surface reflection followed by a series of internal reflections will be observed. Each of the reflected pulses are separated by fixed temporal spacing, which is only determined by the thickness of the sample, the angle of incidence and the refractive index of the sample.

In this model light both reflects off the back and front of the sample and internal reflections are possible. The interaction of an electric field, $E_{0}$,  with such a system in the time domain can be described by the following function:

\begin{eqnarray}
E(t) &=& r_{1}E_{0}(t) + t_{1}t_{2}r_{2}XE_{0}(t-\tau) + t_{1}t_{2}r^{2}_{2}r_{1}X^{2}E_{0}(t-2\tau)\nonumber\\ 
& & + t_{1}t_{2}r^{3}_{2}r^{2}_{1}P^{3}E_{0}(t-3\tau) + ...\nonumber\\
&=& r_{1}E_{0}(t) + t_{1}t_{2}r_{2}X\sum_{m=0}(r_{1}r_{2}X)^{m}E_{0}(t - (m + 1)\tau)\\
\mbox{where:}\nonumber\\
\tau &=& \frac{2dn}{c}\\
X &=& e^{\frac{-2\pi f\kappa d}{c}}\\
d &=& \frac{d_{0}}{\cos{\theta_{1}}}\\
\sin{\theta_{1}} &=& \frac{n_{0}\sin{\theta}}{n}
\label{eq:Single layer time domain Transport function}
\end{eqnarray}

and $r_{1}$ and $t_{1}$ are the reflection and transmission coefficients for light incident on the system from outside as determined from the Fresnel equations, while $r_{2}$ and $t_{2}$ are the reflection and transmission coefficients for light exiting the system. In this equation $\tau$ denotes the time it takes light to travel from a surface back to that surface via reflection, $d_{0}$ is the sample thickness and $n$ is the real refractive index of the material at a given frequency. The frequency of the electric field is denoted by $f$ and $k$ is the extinction coefficient of the material at the given frequency.

Using a Fourier transform, this equation can be rewritten in the frequency domain as follows

\begin{eqnarray}
E(f) &=& E_{0}(f)(r_{1} + t_{1}t_{2}r_{2}A(f)\sum_{m=0}(r_{1}r_{2}A(f))^{m})\label{eq:Single layer frequency domain Transport function}\\
\mbox{where:}\nonumber\\
A(f) &=& e^{\frac{-2\pi f\widetilde{n}d}{c}}\label{eq:AF}\\
\mbox{where:}\nonumber\\
\widetilde{n} &=& n - i\kappa
\label{eq:CR}
\end{eqnarray}

It should be noted that when taking the Fourier transform of the electric field in time, that a time shift will be represented as a linear phase shift in the frequency domain. In equation \ref{eq:AF} the phase shifts present on each of the internal reflections were then combined with the extinction function $X$ and turned into the attenuation function $A$.

\subsection{Complex refractive index extraction}
\label{sub:compref}

The transfer function described in equation \ref{eq:Single layer frequency domain Transport function} will be used in this subsection to build a method to extract the complex refractive index, $\widetilde{n}$ of a sample of interest.

$E_{s}(t)$ and $E_{p}(t)$ have been measured and transformed via FFT to $E_{s}(f)$ and $E_{p}(f)$, \ref{eq:Single layer frequency domain Transport function} can be extended to describe each

\begin{eqnarray}
E_{s}(f) &=& E_{0}(f)(r_{s1} + t_{s1}t_{s2}r_{s2}A(f)\sum_{m=0}(r_{s1}r_{s2}A(f))^{m})\nonumber\\
&=& E_{0}(f)\left(r_{s1} + \frac{t_{s1}t_{s2}r_{s2}A(f)}{1 - r_{s1}r_{s2}A(f)}\right)\label{eq:Single layer frequency domain Transport function s-polarization}\\
E_{p}(f) &=& E_{0}(f)(r_{p1} + t_{p1}t_{p2}r_{p2}A(f)\sum_{m=0}(r_{p1}r_{p2}A(f))^{m})\nonumber\\
&=& E_{0}(f)\left(r_{p1} + \frac{t_{p1}t_{p2}r_{p2}A(f)}{1 - r_{p1}r_{p2}A(f)}\right)\label{eq:Single layer frequency domain Transport function p-polarization}
\end{eqnarray}

The need for the electric field $E_{0}$ to be known is eliminated by taking the ratio between $E_{p}(f)$ and $E_{s}(f)$, thus eliminating the need for a reference sample!

\begin{eqnarray}
H(f) &=&  \frac{r_{p1} + \frac{t_{p1}t_{p2}r_{p2}A(f)}{1 - r_{p1}r_{p2}A(f)}}{r_{s1} + \frac{t_{s1}t_{s2}r_{s2}A(f)}{1 - r_{s1}r_{s2}A(f)}}
\label{eq:Transfer Ratio}
\end{eqnarray}

This theoretical function is then fitted on measured data, using $\widetilde{n}$ as the fit parameter.

To do this, a minimization algorithm known as the Nelder-Mead algorithm was implemented.

The Nelder-Mead algorithm minimizes the error between the theoretical transfer function ratio and the ratio between the measured electric fields by changing $\widetilde{n}$ until a minimum error is achieved.

\subsection{Thickness extraction}
\label{sub: thick}
After the complex refractive index is extracted, as in /ref{sub:compref}, there might be oscillations on it. These oscillations are most likely an artifact caused by an insufficiently accurate sample thickness used in the calculations.

A minimization algorithm can be implemented to minimize these oscillations by altering the thickness of the sample.

\subsection{Complete data extraction method}
\label{sub:datex}
Combining \ref{sub: thick} and \ref{sub:compref} it is possible to create a data extraction method that extracts both the complex refractive index and sample thickness, and accordingly doesn't require perfect knowledge of either.

First it is import to have an initial guess for both the complex refractive index and the thickness of the sample.

For the thickness, a good measure can be taken with a micrometer. For the complex refractive index it is more complicated. A good initial guess can be achieved by applying a Gaussian window on the measured data to only have the outer reflection present. This data can then be processed as if it were the data from a bulk isotropic sample. The complex refractive index produced from this process can then be used as an initial guess for the actual refractive index of the sample.

\begin{figure}[H]
                \begin{center}$
								\begin{array}{cc}
                \includegraphics[scale=0.5]{figs/n-325-20.png}&
                \includegraphics[scale=0.5]{figs/k-325-20.png}
								\end{array}$
								\end{center}
	\caption{The complex refractive index extracted from the data presented in figure \ref{fig:SingTime} via the single layer isotropic data extraction technique}
	\label{fig:SingExt}
\end{figure}

\section{Single layer isotropic medium followed by bulk isotropic sample}
\label{sec:DLM}

An aqueous sample within a container is essentially a two layer system. The first layer is the container, which, if the material is chosen correctly, is a single layer isotropic medium, such as described by the model in section \ref{sec:SLM}. Aqueous solutions have high absorption coefficients for the THz region, thus the second layer can be approached as a bulk isotropic sample, as described by section \ref{sec:BIM}.
%TODO: Add diagram for Double
\begin{figure}[H]
\begin{center}
	 \includegraphics[scale=0.8]{figs/TwoDiag.png}
	 \caption{Diagram depicting light matter interaction with a single layer isotropic medium layered on top of a bulk isotropic sample. Both surface reflections and internal reflections are measurable for the single layer medium, but only surface reflections are measurable for the bulk isotropic sample.}
   \label{fig:TwoDiag}
\end{center}
\end{figure}
%TODO: Add time measurement for Double
%----------------------------------------------------------------------------
\endinput