\chapter{Simulation of THz ellipsomtry measurements}
\label{chp:Simulation}

Simulated data is produced to test the validity of the data extraction techniques discussed in chapter \ref{chp:Analysis}. The strategy is to generate an initial electric field in the time-domain and transform it to frequency domain via a fast Fourier transform (FFT). A transfer function representing the system of interest is then applied to this frequency domain electric field. The resultant electric field is transformed back to the time-domain via an inverse FFT (IFFT). This time-domain data therefore simulates the measured data. This data can then be fed into the data extraction algorithm to test its validity.

\section{Incident Electric Field}
\label{sec:IEF}
As discussed in section \ref{sub: ant}, we make use of a photo-conductive antenna in our setup to generate THz radiation. The electric field generated by this antenna is described by equation \ref{eq:E0sim} and \ref{eq:J0sim}.

The laser we use in our setup to trigger the antenna has a pulse duration, $\tau_{p}$, of $90\,$fs. Low temperature grown GaAs (LT-GaAS), the substrate on which our antennae are printed, has an excitation lifetime, $\tau_{c}$, of $300\,$fs and momentum relaxation time, $\tau_{s}$, of $25\,$fs \cite{Sakai-2005}. The size of the dipole, $l_{e}$, is $20\,\mu$m and the distance to the first collimation optic, $r$, is $45\,$mm. These values were used with equations \ref{eq:E0sim} and \ref{eq:J0sim} to generate the electric field depicted in figure \ref{fig:E0t}. This field is used as our incident field during these simulations.

\begin{figure}[H]
\begin{center}
\includegraphics[scale=0.6]{figs/E0t.png}
\end{center}
\caption[Simulated terahertz electric field in time]{\label{fig:E0t} Time-domain electric field generated via equation \ref{eq:E0sim} for a $20\,\mu$m antenna on LT-GaAs.}
\end{figure}

\section{General transfer function}
\label{sec:LMI}
As the simplest and most general example, a THz pulse incident on a single layer isotropic dielectric medium is initially considered. According to the Fresnel equations, as discussed in section \ref{sec:Fresnel}, an electric field incident on the interface between two media with different refractive indexes is partially transmitted and reflected. There is an initial external reflection from the sample, followed by a train of internal reflections transmitted out of the sample (Etalon effect). Using Snell's laws, section \ref{sec:Snell}, the angle of transmission into the sample can be determined, from this the distance light travels through the sample for each transmitted internal reflection can be calculated. This distance is used to calculate the attenuation due to absorption and phase delay of the electric field as it travels through the sample. Considering an electric field, $E_{0}$ incident on a material
with real refractive index $n(f)$, extinction coefficient $\kappa(f)$ and thickness $d_{0}$, at an angle of incidence $\theta_{0}$, the expected reflected s- and p-polarized electric fields are represented by:

\begin{eqnarray}
E_{s}(t) &=& \widetilde{r}_{s01}E_{s0}(t) + \widetilde{t}_{s01}\widetilde{r}_{s12}\widetilde{t}_{s10}AE_{s0}(t)\nonumber\\& &+ \widetilde{t}_{s01}\widetilde{r}_{s12}^{2}\widetilde{r}_{s10}\widetilde{t}_{s10}A^{2}E_{s0}(t) + ...\\
&=& \widetilde{r}_{s01}E_{s0}(t) + \widetilde{t}_{s01}\widetilde{r}_{s12}\widetilde{t}_{s10}A\sum_{j=0}{(\widetilde{r}_{s10}\widetilde{r}_{12}A)^{j}E_{s0}(t)}\label{eq:tss}\\
E_{p}(t) &=& \widetilde{r}_{p01}E_{p0}(t) + \widetilde{t}_{p01}\widetilde{r}_{p12}\widetilde{t}_{p10}AE_{p0}(t)\nonumber\\& & + \widetilde{t}_{p01}\widetilde{r}_{p12}^{2}\widetilde{r}_{p10}\widetilde{t}_{p10}A^{2}E_{p0}(t) + ...\\
&=& \widetilde{r}_{p01}E_{p0}(t) + \widetilde{t}_{p01}\widetilde{r}_{p12}\widetilde{t}_{p10}A\sum_{j=0}{(\widetilde{r}_{p10}\widetilde{r}_{p12}A)^{j}E_{p0}(t)}\label{eq:tpp}\\
A &=& \exp\left[\frac{-2\pi f \widetilde{n}_{1} d}{c}\right]\\
d &=& \frac{2d_{0}}{\cos{\theta_{1 prop}}}\\
\widetilde{n_{1}} &=& n_{1} - i\kappa_{1}
\end{eqnarray}

where $\widetilde{t}_{s01}$, $\widetilde{t}_{p01}$, $\widetilde{r}_{s01}$, $\widetilde{r}_{p01}$ are the Fresnel transmission and reflection coefficients for the interface between the initial medium and the sample. The transmission and reflection coefficients for the interface between the sample and the initial medium are represented by $\widetilde{t}_{s10}$, $\widetilde{t}_{p10}$, $\widetilde{r}_{s10}$ and $\widetilde{r}_{p10}$, where as $\widetilde{t}_{s12}$, $\widetilde{t}_{p12}$, $\widetilde{r}_{s12}$ and $\widetilde{r}_{p12}$ represent the transmission and reflection coefficients for the interface between the sample and the medium following the sample. These coefficients are calculated using the Fresnel equations (section \ref{sec:Fresnel}). The attenuation and phase delay coefficient (section \ref{sec:Attenuation}) is represented by $A$. The distance that the electric field travels trough the medium is represented by $d$. The complex refractive index of the sample is represented by $\widetilde{n}_{1}$. The angle of propagation, $\theta_{1 prop}$, is calculated by use of equation \ref{eq:prop}.

The complex refractive index is frequency dependent, as is the attenuation and phase delay function (equation \ref{eq:Travel31}), hence it is preferable to work in the frequency domain. Equation \ref{eq:tss} and \ref{eq:tpp} are transformed from the time domain to the frequency domain with a Fourier transform.

\begin{eqnarray}
E_{s}(f) &=& (\widetilde{r}_{s01}(f) + \widetilde{t}_{s01}\widetilde{r}_{s12}\widetilde{t}_{s10}A\sum_{j=0}{(\widetilde{r}_{s10}\widetilde{r}_{s12}A)^{j}})E_{s0}(f)\label{eq:fss}\\
E_{p}(f) &=& (\widetilde{r}_{p01}(f) + \widetilde{t}_{p01}\widetilde{r}_{p12}\widetilde{t}_{p10}A\sum_{j=0}{(\widetilde{r}_{p10}\widetilde{r}_{p12}A)^{j}})E_{p0}(f)\label{eq:fpp}
\end{eqnarray}

%\section{Discretization}
%\label{sec:Disc}

%The implementation of FFTs and IFFTs in the simulation cause discretization errors.%\TODO{ref} 
%These errors occur due to the implementation of discrete Fourier transforms, as opposed to continuous Fourier transforms, and the finite spacing between array elements.%\TODO{ref}

%When performing an IFFT to transform simulated data back to the time domain, if the temporal shift introduced by the transfer function is not an integer multiple of the temporal spacing between elements of the array, it will lead to artifacts in the data.%\TODO{ref}

%In order to avoid discretization errors, we have introduced a correction to the real refractive index applied during the simulation.

%First the frequency dependent time-delay is calculated for the electric field traveling through the medium using equation \ref{eq:timedelay}. The travel time is then rounded to the nearest integer multiple of the step size of the time array. Equation \ref{eq:timedelay} is rewritten to make $n$ the subject, which is then implemented to calculate a corrected refractive index by using the rounded travel time.

%\begin{eqnarray}
%\tau(f) &=& \frac{n_{original}(f)\,d}{c}\label{eq:timedelay}\\
%n(f) &=& \sqrt{\frac{(\tau_{r}(f)c)^{2}}{(2d_{0})^{2}} - n_{0}^{2}\sin^{2}{\theta}}\label{eq:ncor}
%\end{eqnarray}

%where $\tau_{r}(f)$ is $\tau(f)$ rounded to the nearest integer multiple of the time step size of the time array.

\section{Geometric Correction}
\label{sec:GeoCor}

From ray tracing it is found that in the case of internal reflections, each internal reflection takes a shorter route through the setup than expected. In turn, this results in travel time distances between reflected pulses being shorter than expected and needs to be corrected for.

\begin{figure}[H]
\begin{center}
\includegraphics[scale=2.0]{figs/Diag22.png}
\end{center}
\caption[Diagram depicting ray-trace of internal reflection compared to surface reflection]{A ray tracing diagram for calculating the travel time difference between two successive reflections traveling through the setup.}
\end{figure}

This path length distance can be calculated by using the angle of incidence, $\theta$, and the thickness of the sample, $d$. The angle at which light reflects from the second mirror, $30^{\circ}$, and the horizontal separation distance between surfaces, $x_{1}$ and $x_{2}$, are known.

The difference between the two path lengths $\Delta l$ can be calculated from the lengths $r_{1}$, $r_{2}$, $r_{3}$, $r_{4}$ and $\Delta x_{2}$.

\begin{eqnarray}
\Delta r_{1} = r_{1} - r_{2}\\
\Delta r_{3} = r_{3} - r_{4}\\
\Delta l = \Delta r_{1} + \Delta r_{2} - \Delta x_{2}
\label{eq:geoIn}
\end{eqnarray}

For an in-depth explanation on how to solve these value, please refer to Appendix \ref{chp:Geom}. This calculation will be used to derive a correction to the travel time of pulses traveling through the system.

For our system, if a $500\,\mu$m thick undoped silicon sample of with an angle of incidence of $60^{\circ}$ is considered, the calculated correction to the travel distance is $227\,\mu$m. This equates to the internal reflections being detected $0.757\,$ps earlier than expected. This value is introduced as a phase correction when travel times through the sample are calculated. If this were to be ignored, the resultant travel times would be erroneous enough to introduce large errors when compared to measured data, especially as this error is cumulative for each successive internal reflection.

\section{High resistivity silicon simulation example}
\label{sec:SimEx}

The simulated data is compared to measured data to verify its validity. A $500\,\mu$m thick sample with a frequency independent real refractive refractive index of $3.4177$ and a frequency independent absorption coefficient of $0.03\,\text{cm}^{-1}$ was simulated, as these values correspond to the measured optical properties of high resistivity float zone silicon \cite{Li-2008, Jepsen-2007, Grischkowsky1990}. This simulated data was compared to data measured by our setup for a $500\,\mu$m thick high resistivity float zone single crystal silicon sample.

%\begin{figure}[H]
%                \begin{center}$
%								\begin{array}{cc}
%                \includegraphics[scale=0.42]{figs/E1tss}&
%                \includegraphics[scale=0.42]{figs/E1tsm.png}
%								\end{array}$
%								\end{center}
%	\caption{(a) is the normalized THz electric fields in the time-domain generated by our simulation technique. (b) is the normalized THz measured in the time-domain by our setup.}
%	\label{fig:SimVMes}
%\end{figure}

\begin{figure}[H]
\begin{center}
\includegraphics[scale=0.6]{figs/Sim/Etpm.png}
\end{center}
\caption[Measured p-polarized terahertz electric field in time for high resistivity silicon]{\label{fig:EpM} Normalized THz electric field measured in the time-domain by our setup for p-polarized light reflected from a $500\,\mu$m thick high resistivity silicon sample.}
\end{figure}

\begin{figure}[H]
\begin{center}
\includegraphics[scale=0.6]{figs/Sim/Etps.png}
\end{center}
\caption[Simulated p-polarized terahertz electric field in time for high resistivity silicon]{\label{fig:EpS} Normalized simulated THz electric field in the time-domain of p-polarized light reflected from a $500\,\mu$m thick high resistivity silicon sample.}
\end{figure}

\begin{figure}[H]
\begin{center}
\includegraphics[scale=0.6]{figs/Sim/Etsm.png}
\end{center}
\caption[Measured s-polarized terahertz electric field in time for high resistivity silicon]{\label{fig:EsM} Normalized THz electric field measured in the time-domain by our setup for s-polarized light reflected from a $500\,\mu$m thick high resistivity silicon sample.}
\end{figure}

\begin{figure}[H]
\begin{center}
\includegraphics[scale=0.6]{figs/Sim/Etss.png}
\end{center}
\caption[Simulated s-polarized terahertz electric field in time for high resistivity silicon]{\label{fig:EsS} Normalized simulated THz electric field in the time-domain of s-polarized light reflected from a $500\,\mu$m thick high resistivity silicon sample.}
\end{figure}

From figures \ref{fig:EpM} - \ref{fig:EsS} it is found that the simulated data visually correlates well with experimental data. The time-delay between successive pulses correspond with one another and the phase shifts in pulses match expectations, which can be seen in pulses flipping when a $\pi$ phase shift occurs. Lastly, the attenuation in successive pulses correspond visually. 

\begin{figure}[H]
\begin{center}
\includegraphics[scale=0.6]{figs/SimComp_s.png}
\end{center}
\caption[Comparison of measured and simulated p-polarized terahertz electric field in time for high resistivity silicon]{\label{fig:ES_Comp} Comparison between simulated and measured normalized THz electric field in the time-domain for s-polarized light reflected from a $500\,\mu$m thick high resistivity silicon sample.}
\end{figure}

\begin{figure}[H]
\begin{center}
\includegraphics[scale=0.6]{figs/SimComp_p.png}
\end{center}
\caption[Comparison of measured and simulated p-polarized terahertz electric field in time for high resistivity silicon]{\label{fig:EP_comp} Comparison between simulated and measured THz electric field in the time-domain for p-polarized light reflected from a $500\,\mu$m thick high resistivity silicon sample. These electric fields have been normalized with respect to there corresponding s-polarized reflections.}
\end{figure}

By comparing the simulated data with measured data, we found that our angle of incidence is $60.3^{\circ}$. At this angle, both the Fresnel coefficients for s- and  p-polarized light and the travel time between subsequent pulses match very well, as can be seen in figure \ref{fig:ES_Comp} and \ref{fig:EP_comp}.
%A more detailed comparison at this point in time is nonsensical, as the samples investigated here do not exactly correspond to those reported in literature, and hence only qualitative comparisons can be made.
Further analysis of this sample can be found in section \ref{sub: HR-Silicon}.

%----------------------------------------------------------------------------
\endinput