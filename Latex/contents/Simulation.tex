\chapter{Simulation of THz ellipsomtry measurements}
\label{chp:Simulation}

Simulated data is produced to test the validity of the data extraction techniques discussed in chapter \ref{chp:Analysis}. The strategy is to generate an initial electric field in the time-domain and transform it to frequency domain via a FFT. A transfer function representing the system of interest is then applied to this frequency domain electric field. The resultant electric field is then transformed back to the time-domain via an inverse FFT (IFFT). This data can then be fed into the data extraction algorithm to test its validity.

\section{Incident Electric Field}
\label{sec:IEF}
As discussed in section \ref{sub: ant}, we make use of a photo-conductive antenna in our setup to generate THz radiation. The electric field generated by this antenna is described by equation \ref{eq:E0sim} and \ref{eq:J0sim}.

The laser we use in our setup to trigger the antenna has a pulse duration, $\tau_{p}$ of $90\,$fs. Low temperature grown GaAs (LT-GaAS), which is the substrate on which our antennae are printed, has an excitation lifetime, $\tau_{c}$ of $300\,$fs and momentum relaxation time, $\tau_{s}$ of $25\,$fs.\cite{Sakai-2005} The size of the dipole, $l_{e}$, is $20\,\mu$m and the distance to the first collimation optic, $r$, is $45\,$mm. These values were used with equations \ref{eq:E0sim} and \ref{eq:J0sim} to generate the electric field depicted in figure \ref{fig:E0t}:

\begin{figure}[H]
\begin{center}
\includegraphics[scale=0.6]{figs/E0t2.png}
\end{center}
\caption{\label{fig:E0t} Time-domain electric field generated via equation \ref{eq:E0sim} for a $20\,\mu$m antenna on LT-GaAs.}
\end{figure}

\section{General transfer function}
\label{sec:LMI}
In our simulation we consider a single layer isotropic dielectric medium. According to the Fresnel equations, as discussed in section \ref{sec:Fresnel}, an electric field incident on the interface between two media with different refractive indexes is fractionally transmitted and reflected. Thus there is an initial external reflection from the sample, followed by a train of internal reflections transmitted out of the sample. From Snell's laws, covered in section \ref{sec:Snell}, the angle of transmission for the sample is calculated, hence the distance light travels through the sample for each transmitted internal reflection can be calculated. This distance is used to calculate the attenuation of the electric field as it travels through the sample. Considering an electric field, $E_{0}$ incident on a material
with real refractive index $n(f)$, extinction coefficient $k(f)$ and thickness $d_{0}$, at an angle of incidence $\theta_{0}$, the expected reflected s- and p-polarized electric fields are represented by:

\begin{eqnarray}
E_{s}(t) &=& r_{s0}E_{s0}(t) + t_{s0}r_{s2}t_{s1}X^{2}E_{s0}(t-2\tau)\nonumber\\& &+ t_{s0}r_{s2}^{2}r_{s1}t_{s1}X^{4}E_{s0}(t-4\tau) + ...\\
&=& r_{s0}E_{s0}(t) + t_{s0}r_{s2}t_{s1}\sum_{j=0}{(r_{s1}r_{s2}X^{2})^{j}E_{s0}(t - (j+1)2\tau)}\label{eq:tss}\\
E_{p}(t) &=& r_{p0}E_{p0}(t) + t_{p0}r_{p2}t_{p1}X^{2}E_{p0}(t-2\tau)\nonumber\\& & + t_{p0}r_{p2}^{2}r_{p1}t_{p1}X^{4}E_{p0}(t-4\tau) + ...\\
&=& r_{p0}E_{p0}(t) + t_{p0}r_{p2}t_{p1}\sum_{j=0}{(r_{p1}r_{p2}X^{2})^{j}E_{p0}(t - (j+1)2\tau)}\label{eq:tpp}\\
d &=& \frac{d_{0}}{\cos{\theta_{1}}}\\
\theta_{1} &=& \sin^{-1}{\left(\frac{\widetilde{n}_{0}}{\widetilde{n}_{1}}\sin{\theta_{0}}\right)}\\
\widetilde{n} &=& n - i\kappa
\end{eqnarray}

where $t_{s0}$, $t_{p0}$, $r_{s0}$, $r_{p0}$ are the Fresnel transmission and reflection coefficients for the interface between the initial medium and the sample. The transmission and reflection coefficients for the interface between the sample and the initial medium are represented by $t_{s1}$, $t_{p1}$, $r_{s1}$, $r_{p1}$ and $t_{s2}$, $t_{p2}$, $r_{s2}$, $r_{p2}$ represent the transmission and reflection coefficients for the interface between the sample and the medium following the sample. These coefficients are calculated using the Fresnel equations (section \ref{sec:Fresnel}). The attenuation and retardation coefficients (section \ref{sec:Attenuation}) are represented by $X$ and $\tau$. The distance that the electric field travels trough the medium is represented by $d$. The complex refractive indexes of the initial medium, the sample and the medium behind the sample are represented by $\widetilde{n}_{0}$, $\widetilde{n}_{1}$ and $\widetilde{n}_{2}$.

The complex refractive index is frequency dependent, hence it is preferable to work in the frequency domain. Equation \ref{eq:tss} and \ref{eq:tpp} are transformed from the time domain to the frequency domain with a Fourier transform.

\begin{eqnarray}
E_{s}(f) &=& (r_{s0}(f) + t_{s0}r_{s2}t_{s1}A\sum_{j=0}{(r_{s1}r_{s2}A^{2})^{j}})E_{s0}(f)\label{eq:fss}\\
E_{p}(f) &=& (r_{p0}(f) + t_{p0}r_{p2}t_{p1}A\sum_{j=0}{(r_{p1}r_{p2}A^{2})^{j}})E_{p0}(f)\label{eq:fpp}\\
A &=& e^{\frac{-2\pi f\kappa d}{c}}e^{\frac{-2i\pi fnd}{c}}\nonumber\\
&=& e^{\frac{-2i\pi f\widetilde{n}d}{c}}\label{eq:atten}
\end{eqnarray}

The Fourier transform of a function with a time delay, results in said delay transforming to a linear phase shift in the frequency domain, which is applied to derive equation \ref{eq:atten}.

\begin{equation}
F(E(t - \tau)) = E(f)e^{-2i\pi\tau}
\end{equation}

\section{Discretization}
\label{sec:Disc}

The implementation of FFTs and IFFTs in the simulation cause discretization errors. These errors occur due to the implementation of discrete Fourier transforms, as opposed to continuous Fourier transforms, and the finite spacing between array elements.

When performing an IFFT to transform simulated data back to the time domain, if the temporal shift introduced by the transfer function is not an integer multiple of the temporal spacing between elements of the array, it will lead to artifacts in the data.

In order to avoid discretization errors, we have introduced a correction to the real refractive index applied during the simulation.
%TODO: check eqqns, moet dalk tau oor doen
First the frequency dependent time-delay is calculated for the electric field traveling through the medium using equation \ref{eq:Travel21}. The travel time is then rounded to the nearest integer multiple of the step size of the time array. Equation \ref{eq:Travel21} is rewritten to make $n$ the subject, which is then implemented to calculate a corrected refractive index by using the rounded travel time.

\begin{equation}
n(f) = \sqrt{\frac{(\tau_{r}(f)c)^{2}}{(2d_{0})^{2}} - n_{0}^{2}\sin^{2}{\theta}}
\label{eq:ncor}
\end{equation}

where $\tau_{r}(f)$ is $\tau(f)$ rounded to the nearest integer multiple of the time step size of the time array.

\section{Example}
\label{sec:SimEx}

The simulation is compared to measured data to verify its validity. To this end, refractive index data extracted by Li et al \cite{Li-2008} for high resistivity float zone silicon was used in our simulation. This simulated data was compared to data measured by our setup for a $500\,\mu$m thick high resistivity float zone single crystal silicon sample.

%\begin{figure}[H]
%                \begin{center}$
%								\begin{array}{cc}
%                \includegraphics[scale=0.42]{figs/E1tss}&
%                \includegraphics[scale=0.42]{figs/E1tsm.png}
%								\end{array}$
%								\end{center}
%	\caption{(a) is the normalized THz electric fields in the time-domain generated by our simulation technique. (b) is the normalized THz measured in the time-domain by our setup.}
%	\label{fig:SimVMes}
%\end{figure}

\begin{figure}[H]
\begin{center}
\includegraphics[scale=0.6]{figs/Sim/EpM2.png}
\end{center}
\caption{\label{fig:EpM} Normalized THz electric field measured in the time-domain by our setup for p-polarized light reflected from a $500\,\mu$m thick high resistivity silicon sample.}
\end{figure}

\begin{figure}[H]
\begin{center}
\includegraphics[scale=0.6]{figs/Sim/EpS2.png}
\end{center}
\caption{\label{fig:EpS} Normalized simulated THz electric field in the time-domain of p-polarized light reflected from a $500\,\mu$m thick high resistivity silicon sample.}
\end{figure}

\begin{figure}[H]
\begin{center}
\includegraphics[scale=0.6]{figs/Sim/EsM2.png}
\end{center}
\caption{\label{fig:EsM} Normalized THz electric field measured in the time-domain by our setup for s-polarized light reflected from a $500\,\mu$m thick high resistivity silicon sample.}
\end{figure}

\begin{figure}[H]
\begin{center}
\includegraphics[scale=0.6]{figs/Sim/EsS2.png}
\end{center}
\caption{\label{fig:EsS} Normalized simulated THz electric field in the time-domain of s-polarized light reflected from a $500\,\mu$m thick high resistivity silicon sample.}
\end{figure}

From figures \ref{fig:EpM} - \ref{fig:EsS} it is found that the simulated data visually correlates well with experimental data. The time-delay between successive pulses correspond and the phase shifts in pulses match expectations (seen in pulses flipping when a $\pi$ phase shift occurs). Lastly, the attenuation in successive pulses correspond.
%----------------------------------------------------------------------------
\endinput