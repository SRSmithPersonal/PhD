\chapter{Introduction}
\label{chp:Introduction}
Terahertz(THz) radiation lies between far infrared and microwave radiation on the electromagnetic spectrum, and is generally seen as being from $0.3 - 10\,$THz. This area can also be seen as the border between optical and electronic wavelengths.
\paragraph{}
Studies in the THz region are of great importance, as picosecond-timescale processes are very prominent in material sciences \cite{Xuequan2018, doi:10.1063/1.5004194, Neshat2013}.
Traditionally it was difficult to generate and measure THz radiation, as the sources were weak, the wavelengths are long, and ambient black body contamination obscured measured data \cite{Neshat2013}. In recent years, developments in THz sources and detectors have come a long way, making lab based measurements in this spectral region far easier \cite{Neshat2013}. THz radiation is now used in many different fields,  such as the characterization of novel solids, optimization studies for coatings, detection of explosives and bio-hazardous materials and non-invasive imaging, to name but a few \cite{Neshat2013}.
\paragraph{}
In previous studies conducted, it was found that conformational changes in polymers were observable in the THz spectrum \cite{Hoshina2010}.
It has been found that biological polymers, proteins, are identifiable in the THz region and have unique optical properties \cite{Xiaohui2018, Born2009, Novelli2017}. These materials need to be suspended in an aqueous solution to maintain their natural behaviour \cite{Xiaohui2018, Born2009, Novelli2017}. This is a limiting factor for transmission based THz spectroscopy, as THz radiation is strongly absorbed by water \cite{Wu-2018}. In previously performed measurements, high power narrow band THz sources were used as a means to overcome this limitation \cite{Xiaohui2018, Born2009, Novelli2017}.
Alternatively, this limitation can be overcome by working with a reflection based system instead of a transmission system. A reference sample is needed in conventional reflection spectroscopy, as this will be required for eliminating the incident electric field from the data during calculations. This in turn introduces strenuous alignment limitations as the path length error between the sample and reference can not be larger than $10\,\mu$m \cite{Naga}.
In our setup, ellipsometry is used to eliminate the need for a reference sample. Ellipsometry compares the p- and s-polarized light reflected from a sample to determine the sample's optical properties.

Ellipsometry has generally been done in the UV, visible and near infrared spectral regions and has been implemented for use in many fields, such as analysis of thin films, semi-conductive substrates, lithographic products, polymer films, proteins, DNA, TFT films, OLEDs and optical coatings \cite{Neshat2013}.

THz ellipsometers are still a very new development in the spectroscopy world, with several other groups having presented their setups in recent years \cite{Neshat2013}.
The layout and optics of our setup are unique and during the course of this dissertation, several novel data extraction techniques will be presented, each for specific sample types.

In this research a terahertz time-domain ellipsometer has been constructed and several test measurements have been performed and analyzed. This process required the design and manufacturing of components needed for the optical setup, as well as the development of data extraction methods  necessary for extracting information from measured data and simulation software for testing the aforementioned data extraction techniques.

%----------------------------------------------------------------------------
\endinput