\chapter{Experimental setup}
\label{chp:Setup}

Ellipsometry is a powerful spectroscopy technique. This spectroscopy technique is especially powerful for examining thin film samples, as well as optically thick materials. Within the THz spectral region this technique will be useful for examining biological samples in aqueous solution and semiconductor materials. No commercial THz ellipsometers exist, thus performing THz ellipsometry measurements necessitates the design, construction and testing of a THz ellipsometry setup. This included the design and manufacture of several components, which will be expanded upon in the following sections.

\section{Terahertz generation and detection}
\label{sec: Tera}

A large portion of black body radiation at room temperature is within the THz region, thus there is a relatively large background within this spectral region. A coherent source is required to allow for measurements in this high background environment.
Several coherent THz sources, both narrow and broadband, are currently available, such as non-linear crystals, photo-conductive antennae, quantum cascade lasers and directly pumped gas lasers. There are also several detector options for THz detections, the most common of which are opto-electric crystals, photo-conductive antennae, Schottky diodes and bolometers. Our setup uses photo-conductive antennae for both detection and generation. Photo-conductive are broadband coherent THz sources and capable of measuring THz electric fields in time.

\subsection{Photo-conductive antennae as THz emitter}
\label{sub: ant}
A photo-conductive antenna is a dipole antenna that is printed on a photo-conductive substrate. Photo-conductive substrates are semi-conductor materials, which are in a highly resistive state while not excited. An incident femtosecond pulse promotes charge carriers in the substrate to the conduction band. This produces a very short lived conductive state in the substrate, thus only allowing for a single current oscillation in the antenna circuit, when a DC voltage is applied to the circuit, before this substrate returns to a resistive state. The current in the circuit is dependent on the applied voltage, the excitation lifetime of the generated charge carriers, the momentum relaxation time of the generated charge carriers and the amount of charge carriers generated in the substrate. The single current oscillation in turn produces a single oscillation THz pulse that is emitted from the antenna which is dependent on the oscillating current and the size of the dipole and is described by the following equation: \cite{Sakai-2005}

\begin{eqnarray}
E(r,t) &=& \frac{l_{e}}{4\pi\epsilon_{0} c^{2} r}\frac{\partial J(t)}{\partial t}\sin{\theta}\label{eq:E0sim}\\
J(t) &=& \frac{\text{e}\tau_{s}}{m}E_{DC}I_{opt}^{0}\int_{0}^{\infty}e^{-(t-t')^{2}/\tau_{p}^{2}-t'/\tau_{c}}[1-e^{-t'/\tau_{s}}]dt'.
\label{eq:J0sim}
\end{eqnarray}

where $J(t)$ is the current in the dipole, $l_{e}$ the effective length of the dipole, $r$ is the distance between the emitting and receiving antennae and $\theta$ the polar observation angle for the dipole. For the discussed system, $\theta$ is taken to be $90^{\circ}$, since the optical path is perpendicular to the emitter. The carrier lifetime of the substrate is represented by $\tau_{c}$, $\tau_{s}$ is the momentum relaxation time of the substrate, $m$ is the effective mass of the charge carriers, e is the charge of an electron and $E_{DC}$ is the applied bias field. A Gaussian pump pulse with a duration of $2\sqrt{\ln{2}\tau_{p}}$ and intensity of $I_{opt}^{0}$ is used. \cite{Sakai-2005}

\begin{figure}[H]
                \begin{center}$
								\begin{array}{cc}
                \includegraphics[scale=0.5]{figs/Antenna2}&
                \includegraphics[scale=1.0]{figs/AntennaPhoto2.png}
								\end{array}$
								\end{center}
	\caption{(a) is a diagram of a photo-conductive antenna. (b) is a magnified image of a photo-conductive antenna used in our setup.}
	\label{fig:PhoAnt}
\end{figure}

\subsection{Photo-conductive antennae as THz receiver}
\label{sub: antr}
Photo-conductive antennae are used for the detection of THz radiation in our setup. THz detection via a photo-conductive antenna works similarly to THz emission via a photo-conductive antenna (as discussed in section \ref{sub: ant}). When a femtosecond laser pulse is incident on the dipole, charge carriers in the substrate are promoted to the conduction band. A THz electric field incident on the antenna accelerates these generated charge carriers, thus producing a current that can be measured. The induced current is equivocal to the incident electric field, thus by measuring current, the electric field will be measured. Due to the short excitation lifetime of the generated charge carriers, the produced current only represents a small temporal part of the THz electric field. The measured current, and what part of the electric field it represents is a function of the temporal overlap between the femtosecond pulse incident on the dipole and the THz electric field incident on the dipole.
\paragraph{}
The induced electric field is represented by the following equation: \cite{Sakai-2005}:

\begin{equation}
J(t) = \text{e}\mu\int^{\infty}_{-\infty}E(t')N(t'-t)dt
\label{eq:DetectCur}
\end{equation}

where $E(t')$ is the incident THz electric field, $N(t')$ is the number of charge carriers in photo-conductive substrate created by the incident femtosecond laser pulse, e the elementary electric charge and $\mu$ the electron mobility. The generated current only represents a small portion of the incident THz electric field, thus a delay stage is required on the optical path between the femtosecond laser and one of the photo-conductive antennae. 
%TODO: THz-TD -> Entire spectrum
This delay stage is used to change which part of the THz electric-field is measured by the receiving antenna by changing the time the THz electric field is incident on the receiving antenna, relative to the time the femtosecond laser pulse is incident on the receiving antenna, hence allowing for the measurement of the entire THz electric field in time.
\paragraph{}
A limiting factor for THz detection via photo-conductive antennae is the presence of THz resonances in the photo-conductive substrate. LT-GaAs has an absorption band between $5\,-\,10\,$THz, thus making it poorly suited to measurements in this region.

\section{Brewster stacks}
\label{sec: Brew}
Ellipsometric measurements require the electric field incident on the sample to have a pure polarization state.
In our setup we use a pure p-polarized electric field (using the horizontal plane as the plane of reference). A Brewster stack is implemented in our setup to achieve this polarization state. 

\begin{figure}[H]
\begin{center}
	 \includegraphics[scale=0.6]{figs/BrewsterPho.png}
	 \caption{A photo of a high resistivity silicon based Brewster stack we manufactured for use in our setup.}
   \label{fig:BrewPho}
\end{center}
\end{figure}

As discussed in section \ref{sec:Fresnel}, light incident on a medium at the Brewster angle undergoes reflection based losses to its s-polarized component, but not its p-polarized component. 
Implementing multiple layers of the given material at the Brewster angle will hence remove s-polarized light and only leave p-polarized light. This type of structure is known as a Brewster stack.

\begin{figure}[H]
\begin{center}
	 \includegraphics[scale=0.5]{figs/BrewsterDiag3.png}
	 \caption{A diagram of a high resistivity silicon based Brewster stack we designed for use in our setup. An electric field $E_{0}$ is incident on the Brewster stack, depicted as a black arrow. The red lines represent the s-polarized electric field reflected at each interface, $r_{s}E_{\text{current}}$. The blue arrow exiting the system represents $E_{p}$ leaving the Brewster stack.}
   \label{fig:BrewDia}
\end{center}
\end{figure}

It should be noted that the Brewster stack is in a mirror formation. As light propagates through the implemented medium, it travels away from the initial entry path due to the change in propagation direction as described by equation \ref{eq:Snell3}. A mirror unit of the medium is used for every initial unit to correct this walk-off, hence the electric field propagating out of the Brewster stack will propagate along the initial path.
\paragraph{}
The material implemented in our Brewster stack is high resistivity silicon, due to its high refractive index and low absorption coefficient in the THz region (high resistivity silicon has a refractive index of $3.125$ and an absorption coefficient of $0.03\,\text{cm}^{-1}$ in the THz region\cite{Li-2008}). The high refractive index of the material leads to the s-transmission coefficients (equation \ref{eq:FresnelTS}) being relatively low at the Brewster angle ($t_{p_{in}}*t_{p_{out}} = 1.0$ and $t_{s_{in}}*t_{s_{out}} = 0.34$), hence less silicon layers are required in order to achieve a highly pure polarization state (in our Brewster stack four layers are used, which leaves $1.25\%$ of the initial s-polarized electric field). The low absorption coefficient leads to low losses to the pulse as a whole. 

\section{Rotational mount}
\label{sec: rot}

Ellipsometry requires for both the s- and p-polarized electric fields reflected from a sample to be measured. Commercial achromatic half-wave plates are not available for the THz region. A rotational mount was designed to rotate the sample, thus changing the plane of incidence of the sample and hence change the polarization of the electric field incident on the sample.

\begin{figure}[H]
\begin{center}
	 \includegraphics[scale=0.6]{figs/RotateEx.png}
	 \caption{A mock-up of a rotational sample mount to allow for s- and p-polarization THz measurements.}
   \label{fig:RotEx}
\end{center}
\end{figure}

\begin{figure}[H]
\begin{center}
	 \includegraphics[scale=0.6]{figs/RotateDiag.png}
	 \caption{A diagram of a THz optical circuit designed for reflection based measurements. The transfer function for the electric field, $E_{0}$, propagated through the given optical circuit is represented by $T$.}
   \label{fig:RotDiag}
\end{center}
\end{figure}

Figure \ref{fig:RotDiag} represents the optical circuit placed inside the rotational mount. The incident electric field, $E_{0}$ is considered to be in a pure p-polarized state. Rotating the optical circuit by $90^{\circ}$ changes the plane of incidence, thus, from the frame of reference of the optical circuit, the electric field propagating through the circuit is purely s-polarized. In the frame of reference of the rest of the optical circuit, outside of the rotated components, the polarization of the electric field did not change, thus the polarizing components will not need to be adjusted. 

\section{Layout}
\label{sec: Lay}

THz time-domain ellipsometric measurements require a suitable optical circuit for measurements to be performed. This optical circuit can be viewed as two distinct components, the optical circuit for the THz electric field and the optical circuit for the femtosecond laser pulse used to excite the photo-conductive substrate of the antennae.

\begin{figure}[H]
\begin{center}
	 \includegraphics[scale=0.45]{figs/SetupDiag.png}
	 \caption{A diagram of an optical circuit designed for time-domain THz ellipsometry measurements.}
   \label{fig:SetDiag}
\end{center}
\end{figure}

\paragraph{}
The femtosecond laser circuit includes a beam-splitter to divide the femtosecond laser pulse onto two paths, one to the emitting antenna and one to the receiving antenna. The path to the emitting antenna includes a delay stage. This delay stage is used to change at which time the THz electric field is generated  relative to when the receiving antenna is excited, thus changing the temporal overlap between the femtosecond laser pulse and the THz pulse at the receiving antenna and allowing for the THz electric field to be measured in time, as discussed in section \ref{sub: antr}.
\paragraph{}
\begin{figure}[H]
\begin{center}
	 \includegraphics[scale=0.45]{figs/SetupEDiag.png}
	 \caption{A diagram of the THz specific component of the optical circuit depicted in figure \ref{fig:SetDiag}}
   \label{fig:SetEDiag}
\end{center}
\end{figure}
The THz circuit is designed for ellipsometry measurements. The closer the angle of incidence is to the Brewster angle of a sample, the larger the difference is between the reflection coefficients for s- and p-polarized electric fields (equation \ref{eq:FresnelRS} and \ref{eq:FresnelRP}). The accuracy to which the optical parameters for a given material can be extracted correlates to the size of this difference. Our system uses an angle of incidence of $60^{\circ}$, since there is a suitable difference between $r_{s}$ and $r_{p}$ for a large variety of samples at this angle. A mirror array is used to achieve this angle of incidence. This mirror array is mounted on a rotational mount, as discussed in section \ref{sec: rot}. Brewster stacks are implemented as both the polarizer and the analyzer. The polarizer is used to achieve a pure polarization state for the electric field incident on the sample. A pure polarization state for the THz electric field incident on the receiving antenna is achieved via the analyzer.
\endinput