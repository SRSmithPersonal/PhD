\documentclass[12pt,oneside,openany,a4paper,%..... Layout
               afrikaans, english,%.............. Global language selection
               ]{memoir}

 \usepackage[PhD,%.......................... PhD dissertation
             goldenblock,%........................ A5 type block (or a5block or wide)
            ]{usthesis}%.......................... US thesis style with memoir

%
% PLEASE read the USthesis documentation for the class options
% and how to set line and paragraph spacing
%
 \usepackage{float}
 \usepackage{url} 
 \usepackage{caption}
%==== Language setup ================================================
 \usepackage[latin1]{inputenc}%................... Recognizes �, �, etc
 \usepackage{babel}%.............................. Language setup

%==== Math setup ====================================================
 \usepackage{amsmath}%............................ Advanced math (before fonts)
 %\usepackage{amssymb}%............................ AMS Symbol fonts

%==== Font setup (default is Computer Modern) =======================
 \usepackage[T1]{fontenc}%........................ Type 1 fonts
 %\usepackage{fourier}
 \usepackage{textcomp}%........................... Additional text character
 \usepackage{bm}%................................. Bold math symbols (after fonts)

%==== Ref's, Bib's and Nomencl ======================================
 \usepackage{usnomencl}%.......................... List of symbols (in usthesis pack)
 \usepackage[sort&compress]{usbib}%.............................. Bibliography    (in usthesis pack)
    \bibliographystyle{usmeg-n}
    \renewcommand\bibfont{\small}
%\usepackage{citesort}		
%\usepackage{cite}
    %% For usmeg-a, the bib is a list of references. If you
    %% are using usmeg-n comment out the following lines
   % \addto{\captionsafrikaans}{\renewcommand{\bibname}{Lys van Verwysings}}
    %\addto{\captionsenglish}{\renewcommand{\bibname}{List of References}}
%==== Graphics and Color ============================================
\usepackage{graphicx}%........................... Graphicx loaded in usthesis
\usepackage{color}%.............................. Color setup
\usepackage{eso-pic}%............................ Shipout commands for watermark
    \newcommand*{\WaterMark}[2][0.2\paperwidth]{%
        \AddToShipoutPicture*{\AtTextCenter{%
                \parbox[c]{0pt}{\makebox[0pt][c]{%
                    \includegraphics[width=#1]{#2}}}}}}

%==== Local Defs ====================================================
\makeatletter

%
% Please insert user defined commands here
% and NOT in the document itself!
%

\makeatother

%==== TITLE PAGE ====================================================
\title{\bfseries
       \AorE{%-- Afrikaans ------------------------------------------
             Ontwikkeling van 'n spektroskopiese teraherts tyd-verwante ellipsometer\\[1ex]
             \normalfont\small\itshape
             (``Construction of a spectroscopic terahertz time-domain ellipsometer'')
            }{%-- English -------------------------------------------
             Construction of a spectroscopic terahertz time-domain ellipsometer
            }}

\author{S.R.\ Smith}{Shane Raymond Smith}

\degree{\AorE{PhD (Fisika)}{PhD (Physics)}}
       {\AorE{Doktoraat in die Natuurwetenskappe in Laser Fisika}
             {PhD of Science in Laser Physics}}

\address{\AorE{%-- Afrikaans ----------------------------------------
        Departement Fisika,\\
        Universiteit Stellenbosch,\\
        Privaatsak X1, Matieland 7602, Suid Afrika.%
             }{%-- English ------------------------------------------
        Department of Physics,\\
        Stellenbosch University ,\\
        Private Bag X1, Matieland 7602, South Africa.
             }}
\faculty{\AorE{Fakulteit Natuur Wetenskap}%
              {Faculty of Nature Science}}
\supervisor{Dr.\ P.H.\ Neethling} 
\cosupervisor{Prof.\  E.G.\ Rohwer}
\setdate{4}{2020}

%====================================================================
%     MAIN DOCUMENT
%====================================================================
\maxsecnumdepth{subsubsection}
\maxtocdepth{subsubsection}

\begin{document}

%==== Front matter ==================================================
 \frontmatter
 \WaterMark{UScrest-WM}
 \TitlePage

 %\DeclarationSign{\includegraphics[width=2cm]{MySignatureFile}}
 %\DeclarationDate{2010/12/10}
 \DeclarationPage

 \include{frontmatter/Chap-Front}

 \tableofcontents
 \clearpage

 \setcounter{lofdepth}{2}
 \listoffigures
 \clearpage

 %\listoftables
 \clearpage

 \include{frontmatter/Nomencl}

%==== Main document =================================================
\mainmatter
   \setsecnumdepth{subsubsection}
   \numberwithin{equation}{section}
   \numberwithin{figure}{chapter}
   \numberwithin{table}{chapter}

\chapter{Introduction}
\label{chp:Introduction}
Terahertz(THz) radiation lies between far infrared and microwave radiation on the electromagnetic spectrum, and is generally seen as being from $0.3 - 10\,$THz. This area can also be seen as the border between optical and electronic wavelengths.
\paragraph{}
Studies in the THz region are of great importance, as picosecond-timescale processes are very prominent in material sciences \cite{Xuequan2018, doi:10.1063/1.5004194, Neshat2013}.
Traditionally it was difficult to generate and measure THz radiation, as the sources were weak, the wavelengths are long, and ambient black body contamination obscured measured data \cite{Neshat2013}. In recent years, developments in THz sources and detectors have come a long way, making lab based measurements in this spectral region far easier \cite{Neshat2013}. THz radiation is now used in many different fields,  such as the characterization of novel solids, optimization studies for coatings, detection of explosives and bio-hazardous materials and non-invasive imaging, to name but a few \cite{Neshat2013}.
\paragraph{}
In previous studies conducted, it was found that conformational changes in polymers were observable in the THz spectrum \cite{Hoshina2010}.
It has been found that biological polymers, proteins, are identifiable in the THz region and have unique optical properties \cite{Xiaohui2018, Born2009, Novelli2017}. These materials need to be suspended in an aqueous solution to maintain their natural behaviour \cite{Xiaohui2018, Born2009, Novelli2017}. This is a limiting factor for transmission based THz spectroscopy, as THz radiation is strongly absorbed by water \cite{Wu-2018}. In previously performed measurements, high power narrow band THz sources were used as a means to overcome this limitation \cite{Xiaohui2018, Born2009, Novelli2017}.
Alternatively, this limitation can be overcome by working with a reflection based system instead of a transmission system. A reference sample is needed in conventional reflection spectroscopy, as this will be required for eliminating the incident electric field from the data during calculations. This in turn introduces strenuous alignment limitations as the path length error between the sample and reference can not be larger than $10\,\mu$m \cite{Naga}.
In our setup, ellipsometry is used to eliminate the need for a reference sample. Ellipsometry compares the p- and s-polarized light reflected from a sample to determine the sample's optical properties.

Ellipsometry has generally been done in the UV, visible and near infrared spectral regions and has been implemented for use in many fields, such as analysis of thin films, semi-conductive substrates, lithographic products, polymer films, proteins, DNA, TFT films, OLEDs and optical coatings \cite{Neshat2013}.

THz ellipsometers are still a very new development in the spectroscopy world, with several other groups having presented their setups in recent years \cite{Neshat2013}.
The layout and optics of our setup are unique and during the course of this dissertation, several novel data extraction techniques will be presented, each for specific sample types.

In this research a terahertz time-domain ellipsometer has been constructed and several test measurements have been performed and analyzed. This process required the design and manufacturing of components needed for the optical setup, as well as the development of data extraction methods  necessary for extracting information from measured data and simulation software for testing the aforementioned data extraction techniques.

%----------------------------------------------------------------------------
\endinput
\chapter{Background theory}
\label{chp:Theory}

Light that has interacted with a material contains information about the material. The extraction of this data necessitates an understanding of how light and matter interact. Light interacting with dielectric materials undergoes several changes that will be discussed in the following sections.

\section{Polarization}
\label{sec:Pol}

Transverse waves, such as electric fields, oscillate perpendicular to the direction of propagation. The orientation of this oscillation, relative to the plane of incidence, is known as the polarization. The polarization can be broken up into two components, s- and p-polarization, where s- refers to the component of the wave oscillating perpendicular to the plane of incidence and p- refers to the component of the wave oscillating parallel to the plane of incidence.

A phase-delay between the s- and p-polarized components of a transverse wave leads to the oscillations precessing around the central axis of the wave, thus leading to what is known as circular or elliptic polarization.

\section{Fresnel Equations}
\label{sec:Fresnel}

At the interface between two materials with different refractive indexes, a fraction of incident light is reflected and a fraction is transmitted. The ratio of the electric field that is reflected and transmitted is described by the Fresnel equations. The ratio can be broken up into reflection and transmission coefficients. These coefficients are polarization dependent, with different coefficients for the p- and s-polarization.\cite{Driscoll-1978}
%TODO: Add continuety comment
\begin{figure}[H]
\begin{center}
	 \includegraphics[scale=0.8]{figs/FresnelDiag.png}
	 \caption{Diagram depicting the interaction of light at the interface between two media with different refractive indexes. A fraction of incident light is reflected and a fraction is transmitted into the second sample.}
   \label{fig:FresRef}
\end{center}
\end{figure}

\begin{eqnarray}
r_{p} &=& \frac{\widetilde{n}_{i}\cos{\theta_{t}} - \widetilde{n}_{t}\cos{\theta_{i}}}{\widetilde{n}_{i}\cos{\theta_{t}} + \widetilde{n}_{t}\cos{\theta_{i}}}\label{eq:FresnelRP}\\
r_{s} &=& \frac{\widetilde{n}_{i}\cos{\theta_{i}} - \widetilde{n}_{t}\cos{\theta_{t}}}{\widetilde{n}_{i}\cos{\theta_{i}} + \widetilde{n}_{t}\cos{\theta_{t}}}\label{eq:FresnelRS}\\
t_{p} &=& \frac{2\widetilde{n}_{i}\cos{\theta_{i}}}{\widetilde{n}_{i}\cos{\theta_{t}} + \widetilde{n}_{t}\cos{\theta_{i}}}\label{eq:FresnelTP}\\
t_{s} &=& \frac{2\widetilde{n}_{i}\cos{\theta_{i}}}{\widetilde{n}_{i}\cos{\theta_{i}} + \widetilde{n}_{t}\cos{\theta_{t}}}\label{eq:FresnelTS}
\end{eqnarray}

where $r_{p}$ and $t_{p}$ are the reflection and transmission coefficients for the p-polarized component of the the incident electric field and $r_{s}$ and $t_{s}$ are the reflection and transmission coefficients for the s-polarized component of the the incident electric field. The complex refractive index of the first medium is represented by $\widetilde{n}_{i}$ and the complex refractive index of the second medium is represented by $\widetilde{n}_{t}$, while $theta_{i}$ represents the angle at which the wave is incident on the interface between the two media and $theta_{t}$ represents the angle at which the wave propagates through the second medium.

From the Fresnel equations (equations \ref{eq:FresnelRP} - \ref{eq:FresnelTS}) it is found that at a specific angle of incidence, for a given material
the reflection coefficient for p-polarized light becomes $0$ while the reflection coefficient for s-polarized light is a non-zero amount. Thus there will be a loss to s-polarized light transmitted through this material, but not p-polarized light. This angle is known as the Brewster angle.

\begin{equation}
\tan{\theta_{B}} = \frac{n_{2}}{n_{1}}
\label{eq:Brewster}
\end{equation}

where $\theta_{B}$ is the Brewster angle.\cite{Griffiths-2008}

\begin{figure}[H]
\begin{center}
	 \includegraphics[scale=0.6]{figs/BrewsterRef2.png}
	 \caption{The s- and p-reflection coefficients for high resistivity silicon as a function of the angle of incidence.}
   \label{fig:BrewRef}
\end{center}
\end{figure}
It should be noted that where the reflection coefficients are negative, it indicates that the reflected electric field will undergo a $\pi$-phase shift, which can be seen in the time domain as an inversion of the electric field.

\section{Snell's Laws}
\label{sec:Snell}
Snell's laws describe how the direction of propagation changes for light at the interface between two dielectric media.

Snell's 1st law states that the incident, reflected and transmitted wave vectors form a plane. This plane is known as the plane of incidence.

Snell's 2nd law states that the angle at which light reflects off the surface of a material is equal to the angle of incidence.

Snell's 2nd law:
\begin{equation}
\theta_{R} = \theta_{I}
\label{eq:Snell2}
\end{equation}

Snell's 3rd law states that light transmitted through the interface between two dielectric materials with different refractive indexes undergoes a change in propagation direction. The transmitted angle is dependent on the ratio between the refractive indexes of the two materials and the angle at which light is incident on the interface.

Snell's 3rd law:
\begin{eqnarray}
n_{1}\sin{\theta_{I}} &=& n_{2}\sin{\theta_{T}}\\
\theta_{T} &=& \sin^{-1}{\left(\frac{n_{1}}{n_{2}}\sin{\theta_{I}}\right)}
\label{eq:Snell3}
\end{eqnarray}
%\pagebreak

\section{Complex refractive index}
\label{sec:Attenuation}

The complex refractive index describes how light propagates through a medium. The complex refractive index consists of two components, the real refractive index, $n$, and the extinction coefficient, $\kappa$.
\begin{equation}
\widetilde{n}  = n - i\kappa
\label{eq:compref}
\end{equation}
%TODO: def wave vector k
Light traveling through a dielectric medium propagates slower compared to vacuum. The propagation speed is inversely proportional to the real refractive index of the medium. The electric field undergoes attenuation as it propagates through the medium. This attenuation is proportion to the extinction coefficient of the material.
%TODO: check vergelyking vir E field deur medium
\begin{eqnarray}
E_{i}(t) &=& E_{0}e^{i(\omega t - kx)}\label{eq:Travel11}\\
E_{t}(t) &=& E_{0}e^{i(\omega t - kx - \frac{2\pi nfd}{c})}e^{-2\pi\frac{f\kappa d}{c}}\label{eq:Travel21}\\
&=& A\, E_{0}(t)\label{eq:Travel31}\\
A &=& e^{\frac{-2i\pi f\widetilde{n}d}{c}}\label{eq:Travel41}
\end{eqnarray}

In equation \ref{eq:Travel21} $E_{t}$ is the electric field after having traveled through the medium and $E_{0}$ is the initial electric field, $d$ is the distance the light travels through the medium, $n$ is the real refractive index of the medium and $\kappa$ is the extinction coefficient of the medium.


%----------------------------------------------------------------------------
\endinput
\chapter{Experimental setup}
\label{chp:Setup}

Ellipsometry is a powerful spectroscopy technique. This spectroscopy technique is especially powerful for examining thin film samples, as well as optically thick materials. Within the THz spectral region this technique will be useful for examining biological samples in aqueous solution and semiconductor materials. No commercial THz ellipsometers exist, thus performing THz ellipsometry measurements necessitates the design, construction and testing of a THz ellipsometry setup. This included the design and manufacture of several components, which will be expanded upon in the following sections.

\section{Terahertz generation and detection}
\label{sec: Tera}

A large portion of black body radiation at room temperature is within the THz region, thus there is a relatively large background within this spectral region. A coherent source is required to allow for measurements in this high background environment.
Several coherent THz sources, both narrow and broadband, are currently available, such as non-linear crystals, photo-conductive antennae, quantum cascade lasers and directly pumped gas lasers. There are also several detector options for THz detections, the most common of which are opto-electric crystals, photo-conductive antennae, Schottky diodes and bolometers. Our setup uses photo-conductive antennae for both detection and generation. Photo-conductive are broadband coherent THz sources and capable of measuring THz electric fields in time.

\subsection{Photo-conductive antennae as THz emitter}
\label{sub: ant}
A photo-conductive antenna is a dipole antenna that is printed on a photo-conductive substrate. Photo-conductive substrates are semi-conductor materials, which are in a highly resistive state while not excited. An incident femtosecond pulse promotes charge carriers in the substrate to the conduction band. This produces a very short lived conductive state in the substrate, thus only allowing for a single current oscillation in the antenna circuit, when a DC voltage is applied to the circuit, before this substrate returns to a resistive state. The current in the circuit is dependent on the applied voltage, the excitation lifetime of the generated charge carriers, the momentum relaxation time of the generated charge carriers and the amount of charge carriers generated in the substrate. The single current oscillation in turn produces a single oscillation THz pulse that is emitted from the antenna which is dependent on the oscillating current and the size of the dipole and is described by the following equation: \cite{Sakai-2005}

\begin{eqnarray}
E(r,t) &=& \frac{l_{e}}{4\pi\epsilon_{0} c^{2} r}\frac{\partial J(t)}{\partial t}\sin{\theta}\label{eq:E0sim}\\
J(t) &=& \frac{\text{e}\tau_{s}}{m}E_{DC}I_{opt}^{0}\int_{0}^{\infty}e^{-(t-t')^{2}/\tau_{p}^{2}-t'/\tau_{c}}[1-e^{-t'/\tau_{s}}]dt'.
\label{eq:J0sim}
\end{eqnarray}

where $J(t)$ is the current in the dipole, $l_{e}$ the effective length of the dipole, $r$ is the distance between the emitting and receiving antennae and $\theta$ the polar observation angle for the dipole. For the discussed system, $\theta$ is taken to be $90^{\circ}$, since the optical path is perpendicular to the emitter. The carrier lifetime of the substrate is represented by $\tau_{c}$, $\tau_{s}$ is the momentum relaxation time of the substrate, $m$ is the effective mass of the charge carriers, e is the charge of an electron and $E_{DC}$ is the applied bias field. A Gaussian pump pulse with a duration of $2\sqrt{\ln{2}\tau_{p}}$ and intensity of $I_{opt}^{0}$ is used. \cite{Sakai-2005}

\begin{figure}[H]
                \begin{center}$
								\begin{array}{cc}
                \includegraphics[scale=0.5]{figs/Antenna2}&
                \includegraphics[scale=1.0]{figs/AntennaPhoto2.png}
								\end{array}$
								\end{center}
	\caption{(a) is a diagram of a photo-conductive antenna. (b) is a magnified image of a photo-conductive antenna used in our setup.}
	\label{fig:PhoAnt}
\end{figure}

\subsection{Photo-conductive antennae as THz receiver}
\label{sub: antr}
Photo-conductive antennae are used for the detection of THz radiation in our setup. THz detection via a photo-conductive antenna works similarly to THz emission via a photo-conductive antenna (as discussed in section \ref{sub: ant}). When a femtosecond laser pulse is incident on the dipole, charge carriers in the substrate are promoted to the conduction band. A THz electric field incident on the antenna accelerates these generated charge carriers, thus producing a current that can be measured. The induced current is equivocal to the incident electric field, thus by measuring current, the electric field will be measured. Due to the short excitation lifetime of the generated charge carriers, the produced current only represents a small temporal part of the THz electric field. The measured current, and what part of the electric field it represents is a function of the temporal overlap between the femtosecond pulse incident on the dipole and the THz electric field incident on the dipole.
\paragraph{}
The induced electric field is represented by the following equation: \cite{Sakai-2005}:

\begin{equation}
J(t) = \text{e}\mu\int^{\infty}_{-\infty}E(t')N(t'-t)dt
\label{eq:DetectCur}
\end{equation}

where $E(t')$ is the incident THz electric field, $N(t')$ is the number of charge carriers in photo-conductive substrate created by the incident femtosecond laser pulse, e the elementary electric charge and $\mu$ the electron mobility. The generated current only represents a small portion of the incident THz electric field, thus a delay stage is required on the optical path between the femtosecond laser and one of the photo-conductive antennae. 
%TODO: THz-TD -> Entire spectrum
This delay stage is used to change which part of the THz electric-field is measured by the receiving antenna by changing the time the THz electric field is incident on the receiving antenna, relative to the time the femtosecond laser pulse is incident on the receiving antenna, hence allowing for the measurement of the entire THz electric field in time.
\paragraph{}
A limiting factor for THz detection via photo-conductive antennae is the presence of THz resonances in the photo-conductive substrate. LT-GaAs has an absorption band between $5\,-\,10\,$THz, thus making it poorly suited to measurements in this region.

\section{Brewster stacks}
\label{sec: Brew}
Ellipsometric measurements require the electric field incident on the sample to have a pure polarization state.
In our setup we use a pure p-polarized electric field (using the horizontal plane as the plane of reference). A Brewster stack is implemented in our setup to achieve this polarization state. 

\begin{figure}[H]
\begin{center}
	 \includegraphics[scale=0.6]{figs/BrewsterPho.png}
	 \caption{A photo of a high resistivity silicon based Brewster stack we manufactured for use in our setup.}
   \label{fig:BrewPho}
\end{center}
\end{figure}

As discussed in section \ref{sec:Fresnel}, light incident on a medium at the Brewster angle undergoes reflection based losses to its s-polarized component, but not its p-polarized component. 
Implementing multiple layers of the given material at the Brewster angle will hence remove s-polarized light and only leave p-polarized light. This type of structure is known as a Brewster stack.

\begin{figure}[H]
\begin{center}
	 \includegraphics[scale=0.5]{figs/BrewsterDiag3.png}
	 \caption{A diagram of a high resistivity silicon based Brewster stack we designed for use in our setup. An electric field $E_{0}$ is incident on the Brewster stack, depicted as a black arrow. The red lines represent the s-polarized electric field reflected at each interface, $r_{s}E_{\text{current}}$. The blue arrow exiting the system represents $E_{p}$ leaving the Brewster stack.}
   \label{fig:BrewDia}
\end{center}
\end{figure}

It should be noted that the Brewster stack is in a mirror formation. As light propagates through the implemented medium, it travels away from the initial entry path due to the change in propagation direction as described by equation \ref{eq:Snell3}. A mirror unit of the medium is used for every initial unit to correct this walk-off, hence the electric field propagating out of the Brewster stack will propagate along the initial path.
\paragraph{}
The material implemented in our Brewster stack is high resistivity silicon, due to its high refractive index and low absorption coefficient in the THz region (high resistivity silicon has a refractive index of $3.125$ and an absorption coefficient of $0.03\,\text{cm}^{-1}$ in the THz region\cite{Li-2008}). The high refractive index of the material leads to the s-transmission coefficients (equation \ref{eq:FresnelTS}) being relatively low at the Brewster angle ($t_{p_{in}}*t_{p_{out}} = 1.0$ and $t_{s_{in}}*t_{s_{out}} = 0.34$), hence less silicon layers are required in order to achieve a highly pure polarization state (in our Brewster stack four layers are used, which leaves $1.25\%$ of the initial s-polarized electric field). The low absorption coefficient leads to low losses to the pulse as a whole. 

\section{Rotational mount}
\label{sec: rot}

Ellipsometry requires for both the s- and p-polarized electric fields reflected from a sample to be measured. Commercial achromatic half-wave plates are not available for the THz region. A rotational mount was designed to rotate the sample, thus changing the plane of incidence of the sample and hence change the polarization of the electric field incident on the sample.

\begin{figure}[H]
\begin{center}
	 \includegraphics[scale=0.6]{figs/RotateEx.png}
	 \caption{A mock-up of a rotational sample mount to allow for s- and p-polarization THz measurements.}
   \label{fig:RotEx}
\end{center}
\end{figure}

\begin{figure}[H]
\begin{center}
	 \includegraphics[scale=0.6]{figs/RotateDiag.png}
	 \caption{A diagram of a THz optical circuit designed for reflection based measurements. The transfer function for the electric field, $E_{0}$, propagated through the given optical circuit is represented by $T$.}
   \label{fig:RotDiag}
\end{center}
\end{figure}

Figure \ref{fig:RotDiag} represents the optical circuit placed inside the rotational mount. The incident electric field, $E_{0}$ is considered to be in a pure p-polarized state. Rotating the optical circuit by $90^{\circ}$ changes the plane of incidence, thus, from the frame of reference of the optical circuit, the electric field propagating through the circuit is purely s-polarized. In the frame of reference of the rest of the optical circuit, outside of the rotated components, the polarization of the electric field did not change, thus the polarizing components will not need to be adjusted. 

\section{Layout}
\label{sec: Lay}

THz time-domain ellipsometric measurements require a suitable optical circuit for measurements to be performed. This optical circuit can be viewed as two distinct components, the optical circuit for the THz electric field and the optical circuit for the femtosecond laser pulse used to excite the photo-conductive substrate of the antennae.

\begin{figure}[H]
\begin{center}
	 \includegraphics[scale=0.45]{figs/SetupDiag.png}
	 \caption{A diagram of an optical circuit designed for time-domain THz ellipsometry measurements.}
   \label{fig:SetDiag}
\end{center}
\end{figure}

\paragraph{}
The femtosecond laser circuit includes a beam-splitter to divide the femtosecond laser pulse onto two paths, one to the emitting antenna and one to the receiving antenna. The path to the emitting antenna includes a delay stage. This delay stage is used to change at which time the THz electric field is generated  relative to when the receiving antenna is excited, thus changing the temporal overlap between the femtosecond laser pulse and the THz pulse at the receiving antenna and allowing for the THz electric field to be measured in time, as discussed in section \ref{sub: antr}.
\paragraph{}
\begin{figure}[H]
\begin{center}
	 \includegraphics[scale=0.45]{figs/SetupEDiag.png}
	 \caption{A diagram of the THz specific component of the optical circuit depicted in figure \ref{fig:SetDiag}}
   \label{fig:SetEDiag}
\end{center}
\end{figure}
The THz circuit is designed for ellipsometry measurements. The closer the angle of incidence is to the Brewster angle of a sample, the larger the difference is between the reflection coefficients for s- and p-polarized electric fields (equation \ref{eq:FresnelRS} and \ref{eq:FresnelRP}). The accuracy to which the optical parameters for a given material can be extracted correlates to the size of this difference. Our system uses an angle of incidence of $60^{\circ}$, since there is a suitable difference between $r_{s}$ and $r_{p}$ for a large variety of samples at this angle. A mirror array is used to achieve this angle of incidence. This mirror array is mounted on a rotational mount, as discussed in section \ref{sec: rot}. Brewster stacks are implemented as both the polarizer and the analyzer. The polarizer is used to achieve a pure polarization state for the electric field incident on the sample. A pure polarization state for the THz electric field incident on the receiving antenna is achieved via the analyzer.
\endinput
\include{contents/simulation}
\chapter{Data Analysis}
\label{chp:Analysis}

Currently there is a distinct lack of data extraction techniques developed for terahertz time-domain ellipsometry. These techniques are required for extracting material information contained within measured data. The techniques discussed in this chapter focus predominantly on the extraction of the complex refractive index for a given sample from measured data.Three different techniques will be discussed, each with a specific usage case.

\section{Bulk isotropic model}
\label{sec:BIM}
Bulk isotropic materials are optically isotropic dielectric materials, hence not causing depolarization, that have a thickness and optical density that does not allow for measurable internal reflections. Accordingly, for this model only first surface reflections will be considered.

\begin{figure}[H]
\begin{center}
	 \includegraphics[scale=0.8]{figs/BulkDiag.png}
	 \caption{Diagram depicting light matter interaction with a bulk isotropic sample. Only surface reflections are measurable, as internal reflections are not measurable.}
   \label{fig:BulkDiag}
\end{center}
\end{figure}

\begin{figure}[H]
\begin{center}
	 \includegraphics[scale=0.8]{figs/Time-325-1000.png}
	 \caption{Simulated terahertz electric field $E_{0}$ in time and the s- and p-polarized components of said electric field after being reflected from a bulk isotropic sample with a real refractive index of $3.25$ and absorption coefficient of $1000\,\mbox{cm}^{-1}$}
   \label{fig:BulkTime}
\end{center}
\end{figure}

Let us consider the electric field reflected from a bulk isotropic sample. The observed reflected electric field measured in time for the s- and p-polarization can be described as follows

\begin{eqnarray}
E_{s}(t) &=& r_{s}(f)E_{0}(t) \label{eq:Es-bulk Time}\\
E_{p}(t) &=& r_{p}(f)E_{0}(t) \label{eq:Ep-bulk Time}
\end{eqnarray}

where $r_{s}(f)$ and $r_{p}(f)$ are the frequency dependent s- and p-reflection coefficients (equation \ref{eq:FresnelRP} and \ref{eq:FresnelRS}) and $E_{0}$ is the incident electric field.

It should be noted that $r_{s}(f)$ and $r_{p}(f)$ are frequency dependent, hence it is convenient to work in the frequency domain. By performing a fast Fourier transform (FFT) on the data it is converted from the time to the frequency domain, hence the electric field components are rewritten as

\begin{eqnarray}
E_{s}(f) &=& r_{s}(f)E_{0}(f) \label{eq:Es-bulk Frequency}\\
E_{p}(f) &=& r_{p}(f)E_{0}(f) \label{eq:Ep-bulk Frequency}
\end{eqnarray}

Using standard ellipsometric data analysis, it is possible to extract the complex refractive index from this data.\cite{Tompkins-2005}

\begin{eqnarray}
P(f) &=& \frac{E_{p}(f)}{E_{s}(f)}\\
\widetilde{\epsilon}(f) &=& \widetilde{n}_{0}\sin^{2}{\theta}\left[1 + \left(\frac{1-P(f)}{1+P(f)}\right)^{2}\tan^{2}{\theta}\right]
\label{eq:Ellips}\\
\widetilde{n}(f) &=& \sqrt{\widetilde{\epsilon}(f)}
\end{eqnarray}

The angle of incidences is given by $\theta$ in equation \ref{eq:Ellips}, $\widetilde{n}_{0}$ is the complex refractive index of the material surrounding the sample, $\widetilde{\epsilon}(f)$ is the frequency dependent complex dielectric constant of the sample and $\widetilde{n}(f)$ is the frequency dependent complex refractive index of the material.

\begin{figure}[H]
                \begin{center}$
								\begin{array}{cc}
                \includegraphics[scale=0.5]{figs/n-325-1000.png}&
                \includegraphics[scale=0.5]{figs/k-325-1000.png}
								\end{array}$
								\end{center}
	\caption{The complex refractive index extracted from the data presented in figure \ref{fig:BulkTime} via the bulk isotropic data extraction technique}
	\label{fig:BulkExt}
\end{figure}

\section{Single layer isotropic model}
\label{sec:SLM}

Single layer isotropic samples are optically isotropic dielectric materials. These samples have a thickness and optical density that allow for observable internal reflections. This model expands on the model in section \ref{sec:BIM} by incorporating said internal reflections.

\begin{figure}[H]
\begin{center}
	 \includegraphics[scale=0.8]{figs/SingleDiag.png}
	 \caption{Diagram depicting light matter interaction with a single layer isotropic sample. Both surface reflections and internal reflections are measurable.}
   \label{fig:SingleDiag}
\end{center}
\end{figure}

\begin{figure}[H]
\begin{center}
	 \includegraphics[scale=0.8]{figs/Time-325-20.png}
	 \caption{Simulated terahertz electric field $E_{0}$ in time and the s- and p-polarized components of said electric field after being reflected from a single layer isotropic sample with a real refractive index of $3.25$ and absorption coefficient of $20\,\mbox{cm}^{-1}$}
   \label{fig:SingTime}
\end{center}
\end{figure}

\subsection{Transfer Function}
\label{sub: transp}
Transfer functions describe how light travels through a sample. 

When taking a measurement in time of a single layer isotropic system an initial surface reflection followed by a series of internal reflections will be observed. Each of the reflected pulses are separated by fixed temporal spacing, which is only determined by the thickness of the sample, the angle of incidence and the refractive index of the sample.

In this model light both reflects off the back and front of the sample and internal reflections are possible. The interaction of an electric field, $E_{0}$,  with such a system in the time domain can be described by the following function:

\begin{eqnarray}
E(t) &=& r_{1}E_{0}(t) + t_{1}t_{2}r_{2}XE_{0}(t-\tau) + t_{1}t_{2}r^{2}_{2}r_{1}X^{2}E_{0}(t-2\tau)\nonumber\\ 
& & + t_{1}t_{2}r^{3}_{2}r^{2}_{1}P^{3}E_{0}(t-3\tau) + ...\nonumber\\
&=& r_{1}E_{0}(t) + t_{1}t_{2}r_{2}X\sum_{m=0}(r_{1}r_{2}X)^{m}E_{0}(t - (m + 1)\tau)\\
\mbox{where:}\nonumber\\
\tau &=& \frac{2dn}{c}\\
X &=& e^{\frac{-2\pi f\kappa d}{c}}\\
d &=& \frac{d_{0}}{\cos{\theta_{1}}}\\
\sin{\theta_{1}} &=& \frac{n_{0}\sin{\theta}}{n}
\label{eq:Single layer time domain Transport function}
\end{eqnarray}

and $r_{1}$ and $t_{1}$ are the reflection and transmission coefficients for light incident on the system from outside as determined from the Fresnel equations, while $r_{2}$ and $t_{2}$ are the reflection and transmission coefficients for light exiting the system. In this equation $\tau$ denotes the time it takes light to travel from a surface back to that surface via reflection, $d_{0}$ is the sample thickness and $n$ is the real refractive index of the material at a given frequency. The frequency of the electric field is denoted by $f$ and $k$ is the extinction coefficient of the material at the given frequency.

Using a Fourier transform, this equation can be rewritten in the frequency domain as follows

\begin{eqnarray}
E(f) &=& E_{0}(f)(r_{1} + t_{1}t_{2}r_{2}A(f)\sum_{m=0}(r_{1}r_{2}A(f))^{m})\label{eq:Single layer frequency domain Transport function}\\
\mbox{where:}\nonumber\\
A(f) &=& e^{\frac{-2\pi f\widetilde{n}d}{c}}\label{eq:AF}\\
\mbox{where:}\nonumber\\
\widetilde{n} &=& n - i\kappa
\label{eq:CR}
\end{eqnarray}

It should be noted that when taking the Fourier transform of the electric field in time, that a time shift will be represented as a linear phase shift in the frequency domain. In equation \ref{eq:AF} the phase shifts present on each of the internal reflections were then combined with the extinction function $X$ and turned into the attenuation function $A$.

\subsection{Complex refractive index extraction}
\label{sub:compref}

The transfer function described in equation \ref{eq:Single layer frequency domain Transport function} will be used in this subsection to build a method to extract the complex refractive index, $\widetilde{n}$ of a sample of interest.

$E_{s}(t)$ and $E_{p}(t)$ have been measured and transformed via FFT to $E_{s}(f)$ and $E_{p}(f)$, \ref{eq:Single layer frequency domain Transport function} can be extended to describe each

\begin{eqnarray}
E_{s}(f) &=& E_{0}(f)(r_{s1} + t_{s1}t_{s2}r_{s2}A(f)\sum_{m=0}(r_{s1}r_{s2}A(f))^{m})\nonumber\\
&=& E_{0}(f)\left(r_{s1} + \frac{t_{s1}t_{s2}r_{s2}A(f)}{1 - r_{s1}r_{s2}A(f)}\right)\label{eq:Single layer frequency domain Transport function s-polarization}\\
E_{p}(f) &=& E_{0}(f)(r_{p1} + t_{p1}t_{p2}r_{p2}A(f)\sum_{m=0}(r_{p1}r_{p2}A(f))^{m})\nonumber\\
&=& E_{0}(f)\left(r_{p1} + \frac{t_{p1}t_{p2}r_{p2}A(f)}{1 - r_{p1}r_{p2}A(f)}\right)\label{eq:Single layer frequency domain Transport function p-polarization}
\end{eqnarray}

The need for the electric field $E_{0}$ to be known is eliminated by taking the ratio between $E_{p}(f)$ and $E_{s}(f)$, thus eliminating the need for a reference sample!

\begin{eqnarray}
H(f) &=&  \frac{r_{p1} + \frac{t_{p1}t_{p2}r_{p2}A(f)}{1 - r_{p1}r_{p2}A(f)}}{r_{s1} + \frac{t_{s1}t_{s2}r_{s2}A(f)}{1 - r_{s1}r_{s2}A(f)}}
\label{eq:Transfer Ratio}
\end{eqnarray}

This theoretical function is then fitted on measured data, using $\widetilde{n}$ as the fit parameter.

To do this, a minimization algorithm known as the Nelder-Mead algorithm was implemented.

The Nelder-Mead algorithm minimizes the error between the theoretical transfer function ratio and the ratio between the measured electric fields by changing $\widetilde{n}$ until a minimum error is achieved.

\subsection{Thickness extraction}
\label{sub: thick}
After the complex refractive index is extracted, as in /ref{sub:compref}, there might be oscillations on it. These oscillations are most likely an artifact caused by an insufficiently accurate sample thickness used in the calculations.

A minimization algorithm can be implemented to minimize these oscillations by altering the thickness of the sample.

\subsection{Complete data extraction method}
\label{sub:datex}
Combining \ref{sub: thick} and \ref{sub:compref} it is possible to create a data extraction method that extracts both the complex refractive index and sample thickness, and accordingly doesn't require perfect knowledge of either.

First it is import to have an initial guess for both the complex refractive index and the thickness of the sample.

For the thickness, a good measure can be taken with a micrometer. For the complex refractive index it is more complicated. A good initial guess can be achieved by applying a Gaussian window on the measured data to only have the outer reflection present. This data can then be processed as if it were the data from a bulk isotropic sample. The complex refractive index produced from this process can then be used as an initial guess for the actual refractive index of the sample.

\begin{figure}[H]
                \begin{center}$
								\begin{array}{cc}
                \includegraphics[scale=0.5]{figs/n-325-20.png}&
                \includegraphics[scale=0.5]{figs/k-325-20.png}
								\end{array}$
								\end{center}
	\caption{The complex refractive index extracted from the data presented in figure \ref{fig:SingTime} via the single layer isotropic data extraction technique}
	\label{fig:SingExt}
\end{figure}

\section{Single layer isotropic medium followed by bulk isotropic sample}
\label{sec:DLM}

An aqueous sample within a container is essentially a two layer system. The first layer is the container, which, if the material is chosen correctly, is a single layer isotropic medium, such as described by the model in section \ref{sec:SLM}. Aqueous solutions have high absorption coefficients for the THz region, thus the second layer can be approached as a bulk isotropic sample, as described by section \ref{sec:BIM}.
%TODO: Add diagram for Double
\begin{figure}[H]
\begin{center}
	 \includegraphics[scale=0.8]{figs/TwoDiag.png}
	 \caption{Diagram depicting light matter interaction with a single layer isotropic medium layered on top of a bulk isotropic sample. Both surface reflections and internal reflections are measurable for the single layer medium, but only surface reflections are measurable for the bulk isotropic sample.}
   \label{fig:TwoDiag}
\end{center}
\end{figure}
%TODO: Add time measurement for Double
%----------------------------------------------------------------------------
\endinput
\include{contents/results}
\include{contents/conclusion}
%==== Appendices ====================================================
\appendix
\appendixpage\relax

\include{contents/Knife-Edge}
\chapter{Alignment}
\label{chp:Align}

\section{Rotational mount alignment}
\label{RotAlign}

The rotational mount, as described in section \ref{sec: rot}, is an integral component of our THz ellipsometry system. This system will house the stage on which our sample and several mirrors will be mounted. It is paramount that this mount is aligned as optimally as possible, as any miss-alignment it incurs will introduce significant errors to measured data.

Several steps were taken to align the rotational mount. A secondary optical route from the femtosecond laser to the receiving antenna was constructed. This allows the system to operate in transmission when needed. %TODO add diagram

When the antennae are removed from their mounts, the femtosecond laser travels from the one mount to the other via the current THz optical path. The femtosecond laser was used to find an optimal placement and height for the antenna mounts, such that the light travels through the center of the mount. The mount was then removed, the antennas inserted and aligned in a transmission configuration.

By using a fast oscillating stage on the path to the transmitting antenna, the pulse can be measured on an oscilloscope and changes made by alterations to the alignment of the mount can be viewed in real time. The stage is then placed back in the THz optical path and manipulated until optimal throughput is achieved.

Once the Rotational mount is placed in the system, the same process is used to find the optimal mounting position for the two Brewster stacks.

\section{Ellipsometric stage and THz antenna alignment}
\label{chp:EllipsAlign}
The THz ellipsometric system is sensitive to miss-alignment. An iterative process for aligning the setup was developed. 

A HeNe laser is employed for aligning the lenses, mirrors and sample on the plate mounted on the rotational mount. The use of the HeNe laser requires that the Brewster stacks be removed, as the laser light cannot pass through the silicon.

\begin{figure}[H]
\begin{center}
	 \includegraphics[scale=2.0]{figs/HeNe2.png}
	 \caption[Diagram of HeNe alignment laser optical path]{A diagram of the optical path traveled by the HeNe alignment laser.}
   \label{fig:HeNe}
\end{center}
\end{figure}

A flip-down mirror is used to couple the HeNe laser light onto the THz optical path. This eases the change-over between the use of the alignment laser and the THz pulse in the optical path. Several apertures are used to ensure the HeNe laser is coupled straight in and through the rotational mount, as can be seen in figure \ref{fig:HeNe}. The lenses used in our THz system are made from the polymer TPX. This material allows for both visible and THz radiation to pass through it, thus the HeNe laser is used to align these lenses.

A set of measurements, with the rotational mount set to $0^{\circ}$, $90^{\circ}$, $180^{\circ}$ and $270^{\circ}$ respectively, with a silver mirror as the sample is taken via the THz system. These measurements are used to determine whether the THz pulse is traveling straight through the ellipsometric system. If not, the alignment of the antennae must be corrected such that the pulse travels straight through the ellipsometric system. An example of measured results before and after this alignment adjustment can be seen in figure \ref{fig:Align_center}.

\begin{figure}[H]
                \begin{center}$
								\begin{array}{cc}
                \includegraphics[scale=0.4]{figs/Align_poor}&
                \includegraphics[scale=0.4]{figs/Align_better.png}
								\end{array}$
								\end{center}
	\caption[Example of centering alignment]{(a) is a THz time-domain measurement which is poorly centered. (b) is a repeat of the measurement performed for (a), after the alignment of the photo-conductive antennae was improved regarding radiation going straight through the rotational mount.}
	\label{fig:Align_center}
\end{figure}

Next, using a known sample, measurements for p- and s-polarized light reflected from the sample are performed. These measured time-domain electric fields are then compared with electric field produced via simulation, as described in section \ref{chp:Simulation}. This comparison allows for the determination of the angle of incidence, as well as whether the sample has been mounted skew. 

The angle of incidence is determined by comparing the ratio between the first pulse measured for the s- and p- electric fields, with that of the simulated electric fields. This angle of incidence is applied to the simulation and compared to the measured fields. If the sample has been mounted skew, this will be evidenced by poor overlap between simulated and measured arrival times for pulses beyond the initial pulse. In figure \ref{fig:Align_skew} it can be seen how the secondary pulses drift from the expected arrival time when the sample has been inserted skew.

\begin{figure}[H]
                \begin{center}$
								\begin{array}{cc}
                \includegraphics[scale=0.4]{figs/Align_skew}&
                \includegraphics[scale=0.4]{figs/Align_straight.png}
								\end{array}$
								\end{center}
	\caption[Example of straightening the sample]{(a) is a THz time-domain measurement taken of silicon with the sample inserted skew and the simulated expected electric fields. (b) is a repeat of the measurement performed for (a), after the sample was straightened and the system was realigned. The expected simulated electric field is also presented.}
	\label{fig:Align_skew}
\end{figure}

The sample used to align the setup was high resistivity, undoped single crystal silicon, which has been measured to have a flat real refractive index of $3.4177$ and absorption coefficient of $0.03\,\text{cm}^{-1}$ in the THz region \cite{Li-2008, Jepsen-2007, Grischkowsky1990}.

If the sample mount is skew, or the angle of incidence needs to be altered, said alteration is applied and the process of aligning the ellipsometric stage components must be repeated.

\endinput
\chapter{Calculation of the geometric correction}
\label{chp:Geom}

\begin{figure}[H]
\begin{center}
\includegraphics[scale=2.0]{figs/Diag22.png}
\end{center}
\caption[Diagram depicting ray-trace of internal reflection compared to surface reflection]{A ray tracing diagram for calculating the travel time difference between two successive reflections traveling through the setup.}
\end{figure}

This system can be divided up into two components, a system encompassing $r_{1}$ and $r_{2}$ and a system focusing on $r_{3}$ and $r_{4}$.

$\Delta x_{1}$ is calculated by using Snell's laws.

\begin{eqnarray}
\theta_{1} &=& \sin^{-1}(\frac{n_{0}\sin(\theta)}{n_{1}})\\
\Delta x_{1} &=& 2d\tan(\theta_{1})
\label{eqn:Geo0}
\end{eqnarray}

Via construction, a set of similar triangles with $r_{1}$ and $r_{2}$ as parallel sides is drawn.

\begin{figure}[H]
\begin{center}
\includegraphics[scale=1.2]{figs/Sec1-2.png}
\end{center}
\caption[First section of the exiting beam path used to calculate a geometric correction for subsequent beams.]{First part of the construction used to solve the geometric error introduced to subsequent beams.}\label{fig:GeoSec-1}
\end{figure}

$r_{1}$ and $r_{2}$ can be solved by making use of the construction shown in figure \ref{fig:GeoSec-1}. $\Delta K$ can also be extracted from this sketch. 

\begin{eqnarray}
\phi &=& 90^{\circ} - \theta\\
\beta &=& 75^{\circ} + \frac{\phi}{2}\\
AE &=& x_{1} - \frac{x_{1}\tan(\phi)}{\tan{\beta}}\\
AC &=& AE - \Delta x_{1}\\
r_{1} &=& \frac{x_{1}}{\cos(\phi)}\\
K_{1} &=& \sqrt{r_{1}^{2} + AE^{2} - 2r_{1}\,AE\cos(\phi)}\\
K_{2} &=& \frac{K_{1}(AC + 1)}{\Delta x_{1}}\\
r_{2} &=& \sqrt{K_{2}^{2} + AC^{2} - 2K_{2}\,AC\cos{\alpha}}\\
\Delta K &=& K1 - K2
\label{eqn:geo1}
\end{eqnarray}

$r_{3}$ and $r_{4}$, and subsequently $\Delta x_{2}$, can be solved by via a second construction of two similar triangles, with $r_{3}$ and $r_{4}$ as similar sides.

\begin{figure}[H]
\begin{center}
\includegraphics[scale=1.2]{figs/Sec2-2.png}
\end{center}
\caption[Second section of the exiting beam path used to calculate a geometric correction for subsequent beams.]{Second part of the construction used to solve the geometric error introduced to subsequent beams.}\label{fig:GeoSec-2}
\end{figure}

\begin{eqnarray}
\Delta y_{2} &=& dK\sin{\beta}\\
x_{p} &=& dK\cos(\beta) + \frac{dy_{2}}{\tan(30^{\circ})}\\
y_{1} &=& (x_{2} + x_{p})\tan(30^{\circ})\\
x_{3} &=& \frac{y_{1}}{\tan(75^{\circ})}\\
B &=& \frac{y_{1}}{\sin(75^{\circ})}\\
A &=& x_{3} + x_{p} + x_{2}\\
C &=& \frac{x_{2} + x_{p}}{\cos(30^{\circ})}\\
\Delta f &=& \frac{x_{p}B}{(A - x_{p})(1 + \frac{x_{p}}{A - x_{p}})}\\
\Delta x_{2} &=& \Delta f\cos(75^{\circ})\\
r_{3} &=& C - \frac{dy_{2}}{\sin(30^{\circ})}\\
a &=&  A - x_{p}\\
b &=&  B - \Delta f\\
r_{4} &=& \sqrt{a^{2} + b^{2} - 2a\,b\cos(75^{\circ})}\\
\Delta r_{1} &=& r_{1} - r_{2}\\
\Delta r_{2} &=& r_{3} - r_{4}\\
\Delta l &=& \Delta r_{1} + \Delta r_{2} - \Delta x_{2}
\label{eq:eqn:geo2}
\end{eqnarray}

$\Delta l$ represents the total change in the path length of the second ray through the optical setup and is used to calculate how much earlier the the internal reflections will be measured than they would be expected to if only the travel distance through the sample were to be considered.

\endinput

%==== Bibliography acro's & Index ===================================
\backmatter

\bibliography{backmatter/Bib}

\end{document}
